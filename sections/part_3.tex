\documentclass[../constitution.tex]{subfiles}
\begin{document}

\part{MEMBERS} \label{part-3-members}


\section{Membership} \label{division-1-membership}

\chcomment{Variance to model rules - model rules have separate sections for "eligibility for membership" and "applying for membership".}

\note{
  (CAP3-Community-Shed)
  
  For the Association to be eligible for deductible gift recipient (DGR) status as a community shed, the Association must have open membership, as per the definition of a community shed in the \textit{Income Tax Assessment Act 1997 (Cth)}, Division 995 \textit{Definitons} ---

  \vspace{12pt}

  \textbf{community shed} means a public institution that satisfies all of the following requirements:

  (a)	the institution's dominant purposes are advancing mental health and preventing or relieving social isolation;

  (b)	the institution seeks to achieve those purposes principally by providing a physical location where it supports individuals to undertake activities, or work on projects, in the company of others;

  (c)	either:

  (i)	there are no particular criteria for membership of the institution; or

  (ii)	the criteria for membership of the institution relate only to an individual's gender or Indigenous status (in that membership is, for cultural reasons, open only to *Indigenous persons) or both.

  \vspace{12pt}

  The Australian Tax Office \href{https://www.ato.gov.au/Non-profit/Getting-started/In-detail/Types-of-DGRs/Community-sheds/}{describes open membership as} ---


  A community shed must be open to the community to join and generally not impose criteria restricting membership based on matters such as age, ethnicity or background.
  
  Rejecting an application for arbitrary reasons will not constitute open membership.

  Membership may only be restricted in relation to gender or indigenous heritage or both. There are a small number of other exceptional reasons for restricting membership. For example:

  \begin{itemize}
  \item age restrictions in your state or territory
  \item capacity reached by the shed
  \item failing a working with children check required by the premises.
  \end{itemize} 

  Community sheds that are incorporated often adopt model rules provided by their state or territory regulator. Model rules contain clauses that describe how new members can join your organisation.

  In some cases, rules require new members to be nominated by a current member and approved by a committee. Nomination and approval rules will meet open membership requirements where the nomination and approval process results in all new members being approved for membership without restriction or discrimination.

  To meet the open membership requirements, you must have a policy and process in place that clearly demonstrates all new members are nominated and approved without exception. This should be reflected in your governing rules.
}


\subsection{Applying for Membership} \label{applying-for-membership}

\begin{enumerate}
\item Any person who \chdeleted[id=CAP3-Community-Shed]{is at least 18 years of age and} supports the objects or purposes of the Association is eligible to apply to become a member. \label{member-is-eligibile}
\item A person who wishes to become a member of the Association must submit an application to the \chreplaced[id=proofing]{Association}{Committee}. \label{member-must-submit-application}
\item \chreplaced[id=CAP3-Community-Shed]{The applicant must specify in the application the class of membership, if there is more than one, to which the application relates.}{The applicant must specify in the application the class of membership.}
\item \chadded[id=CAP3-Community-Shed]{The applicant must meet the eligibility criteria, defined in the by-laws, for the class of membership they are applying for.} \label{eligible-for-class-of-membership}
\item \chadded[id=CAP3-Community-Shed]{An individual who has not reached the age of 18 years is not eligible to apply for a class of membership that confers full voting rights.} \label{eligible-for-voting-rights}
\item \chadded[id=CAP3-Community-Shed]{An individual who has not reached the age of 18 years may only apply for a class of membership which specifically accepts individuals under the age of 18 years.} \label{members-under-18-years} \chcomment{e.g. a class of membership for high school students or first year university students who aren't 18 years old yet.}
\end{enumerate}

\subsection{Dealing with Membership Applications}\label{dealing-with-membership-applications}

\begin{enumerate}
\item The committee must consider each application for membership of the Association and decide whether to accept or reject the application. \chcomment{Variance to model rules: subrule about considering applications "in the order they are recieved" is deleted.}
\item The committee may delay its consideration of an application if the committee considers that any matter relating to the application needs to be clarified by the applicant or that the applicant needs to provide further information in support of the application.
\item The committee must not accept an application unless the applicant --- \label{membership-must-not-be-accepted}
  \begin{enumerate}
  \item is eligible under rule \ref{member-is-eligibile}; and
  \item has applied under rule \ref{member-must-submit-application}\chadded[id=CAP3-Community-Shed]{; and}.
  \item \chadded[id=CAP3-Community-Shed]{is eligible for the class of membership they have applied for under subrules \ref{eligible-for-class-of-membership}, \ref{eligible-for-voting-rights}, and \ref{members-under-18-years}.}
  \end{enumerate}

\item \chadded[id=CAP3-Community-Shed]{The committee may only reject applications for the following reasons ---}
  \begin{enumerate}
  \item \chadded[id=CAP3-Community-Shed]{the committee cannot accept the application because of subrule \ref{membership-must-not-be-accepted}; or}
  \item \chadded[id=CAP3-Community-Shed]{the committee determines that the Association's overall membership capacity has been reached; or}
  \item \chadded[id=CAP3-Community-Shed]{the committee determines that the Association's membership capacity for a particular class of membership has been reached; or}
  \item \chadded[id=CAP3-Community-Shed]{the applicant has previously been expelled from the Association under rule \ref{suspension-or-expulsion}; or}
  \item \chadded[id=CAP3-Community-Shed]{the applicant has previously ceased to be a member under subrule \ref{member-fails-to-pay-fee}; or}
  \item \chadded[id=CAP3-Community-Shed]{the applicant has acted detrimentally to the interests of the Association}; or
  \item \chadded[id=CAP4-Corporate-Members]{if the applicant is a body corporate - for any reason.}
  \end{enumerate}

\item The committee must notify the applicant of the committee's decision to \chadded[id=CAP3-Community-Shed]{accept or} reject an application as soon as practicable after making the decision.

\item If the committee rejects the application, the committee is not required to give the applicant its reasons for doing so.
\end{enumerate}



\subsection{Becoming a member} \label{becoming-a-member}


An applicant for membership of the Association becomes a member when ---

\begin{enumerate}
\def\labelenumi{\alph{enumi})}
\setcounter{enumi}{0}
\item \chadded[id=CAP3-Community-Shed]{the committee accepts the application; and}\chcomment{Revert to language from the Model Rules: "the committee accepts the application."}
\item the applicant pays any membership fees payable to the Association under rule \chreplaced[id=proofing]{\ref{membership-fees}}{\ref{member-must-pay-fee}}\chreplaced[id=proofing]{.}{; and}
\item \chdeleted[id=proofing]{the committee accepts the application at a committee meeting or as statedin the by-laws.}
\end{enumerate}



\subsection{Classes of membership} \label{classes-of-membership}

\begin{enumerate}
\item The Committee may make multiple classes of ordinary membership and associate membership and may make individual arrangements for membership. 
\item A person \chdeleted[id=proofing]{or entity} can only hold one class of membership. \chcomment{Note the definition of "member" already includes body corporates.}
\item An ordinary member has full voting rights and any other rights conferred on members by these rules or approved by resolution at a general meeting or determined by the committee. \label{ordinary-member-rights}
\item An associate member has the rights referred to in \chreplaced[id=proofing]{subrule}{rule} \ref{ordinary-member-rights} --- \label{associate-member-rights}
  \begin{enumerate}
  \item other than full voting rights;
  \item and rights restricted under a particular class of associate membership.
  \end{enumerate}
\item The Committee may limit the number of members of any class of membership and associate membership.
\item If a particular class of membership is in use, the Association does not have the right except at a General Meeting to change a class of \chadded[id=proofing]{ordinary} membership to an \chreplaced[id=proofing]{class of associate membership}{Associate Class} or vice versa.
\end{enumerate}


\subsection{When Membership Ceases} \label{when-membership-ceases}

\begin{enumerate}

\item A person ceases to be a member when any of the following takes place ---

  \begin{enumerate}
  
  \item for a member who is an individual, the individual dies;
  \item \chadded[id=CAP4-Corporate-Members]{for a member that is a body corporate, the body corporate is wound up;} \chcomment{From Model Rules}
  \item the person resigns from the Association under rule \ref{resignation};
  \item the person is expelled from the Association under rule \ref{suspension-or-expulsion};
  \item the person ceases to be a member under rule \ref{member-fails-to-pay-fee}
  \end{enumerate}
\item The Association must keep a record, for at least one year after a person ceases to be a member, of ---

  \begin{enumerate}
  
  \item the date on which the person ceased to be a member; and
  \item the reason why the person ceased to be a member.
  \end{enumerate}
\end{enumerate}


\subsection{Resignation} \label{resignation}

\begin{enumerate}

\item A member may resign from membership of the Association by giving written notice of the resignation to the Secretary or other methods described in the by-laws. \chcomment{Variance to model rules: "Other methods described in the by-laws."}
\item The resignation takes effect ---
  \begin{enumerate}
  \item when the Secretary receives the notice; or
  \item if a later time is stated in the notice, at that later time; or
  \item when described in the by-laws. \chcomment{Variance to model rules: "when described in the by-laws."}
  \end{enumerate}
\item A person who has resigned from membership of the Association remains liable for any fees that are owed to the Association (the \textbf{owed amount}) at the time of resignation.
\item The owed amount may be recovered by the Association in a court of competent jurisdiction as a debt due to the Association.
\end{enumerate}


\subsection{Rights not Transferable} \label{rights-not-transferable}

The rights of a member are not transferable and end when membership ceases.


\section{Membership fees} \label{division-2-membership-fees}

\subsection{Membership fees} \label{membership-fees}

\begin{enumerate}


\item \label{fees-set-by} \chadded[id=CAP7-Membership-Fees]{The entrance fee (if any) and the membership fee (if any) for membership of the Association must be;}
\begin{enumerate}
  \item \chadded[id=CAP7-Membership-Fees]{determined by the committee; and}
  \item \chadded[id=CAP7-Membership-Fees]{approved at a general meeting.}
\end{enumerate}

\item \label{fees-periods} \chadded[id=CAP7-Membership-Fees]{The committee must determine the acceptable schedules for payment of membership fees.}

\note{e.g. monthly, yearly, lifetime.}

\item \chadded[id=CAP7-Membership-Fees]{The fees determined under subrules \ref{fees-set-by} and \ref{fees-periods} may be different for different classes of membership.}

\item \chdeleted[id=CAP7-Membership-Fees]{The committee must determine the entrance fee (if any) and the period of the membership fee (if any) to be paid for membership of the Association.}
\item \chdeleted[id=CAP7-Membership-Fees]{The membership fee and payment schedule will be fixed by the Management Committee, subject to review by the members at a general meeting.}
\item \chdeleted[id=CAP7-Membership-Fees]{All members must pay the membership fees on the schedule set by the Management Committee.}
\item A member must pay the membership fee to the treasurer, or another person authorised by the committee to accept payments, by the date (the \textbf{due date}) determined by the committee. \label{member-must-pay-fee}
\item If a member has not paid the membership fee within the period of 3 months after the due date, the member ceases to be a member on the expiry of that period. \label{member-fails-to-pay-fee}
\item If a person who has ceased to be a member under \chreplaced[id=proofing]{subrule}{rule} \ref{member-fails-to-pay-fee} offers to pay the \chreplaced[id=CAP7-Membership-Fees
]{membership fee for the relevant class of membership}{appropriate membership fee relevant to their membership level} after the period referred to in that \chreplaced[id=proofing]{subrule}{rule} has expired ---

  \begin{enumerate}
  
  \item the committee may, at its discretion, accept that payment; and
  \item if the payment is accepted, the person's membership is reinstated from the date the payment is accepted.
  \end{enumerate}
\end{enumerate}

\hypertarget{division-3-register-of-members}{%
\section{Register of members}\label{division-3-register-of-members}}

\hypertarget{register-of-members}{%
\subsection{Register of members}\label{register-of-members}}

\begin{enumerate}

\item The secretary, or another person authorised by the committee, is responsible for the requirements imposed on the Association under section 53 of the Act to maintain the register of members and record in that register any change in the membership of the Association.
\item In addition to the matters referred to in section 53(2) of the Act, the register of members must include the class of membership \chadded[id=proofing]{(if applicable)} to which each member belongs and the date on which each member becomes a member.
\item The register of members must be kept at \chadded[id=proofing]{a} place determined by the committee. \chcomment{Variance to model rules: The model rules presume this will be a physical book, which would be usually kept at the secretary's place of residence.}
\item A member who wishes to inspect the register of members must contact the secretary to make the necessary arrangements.
\item If ---

  \begin{enumerate}
  \item a member inspecting the register of members wishes to make a copy of, or take an extract from, the register under section 54(2) of the Act; or
  \item a member makes a written request under section 56(1) of the Act to be provided with a copy of the register of members,

  \end{enumerate}
  the committee may require the member to provide a statutory declaration setting out the purpose for which the copy or extract is required and declaring that the purpose is connected with the affairs of the Association.
\end{enumerate}

\end{document}