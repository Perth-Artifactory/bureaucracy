\documentclass[../constitution.tex]{subfiles}
\begin{document}

\part{MEMBERS} \label{part-3-members}


\section{Membership} \label{division-1-membership}

\subsection{Applying for Membership} \label{applying-for-membership}

\begin{enumerate}
  \item Any person who supports the objects or purposes of the Association is eligible to apply to become a member. \label{member-is-eligibile}
  \item A person who wishes to become a member of the Association must submit an application to the Association. \label{member-must-submit-application}
  \item The applicant must specify in the application the class of membership, if there is more than one, to which the application relates.
  \item The applicant must meet the eligibility criteria, defined in the by-laws, for the class of membership they are applying for. \label{eligible-for-class-of-membership}
  \item An individual who has not reached the age of 18 years is not eligible to apply for a class of membership that confers full voting rights. \label{eligible-for-voting-rights}
  \item An individual who has not reached the age of 18 years may only apply for a class of membership which specifically accepts individuals under the age of 18 years. \label{members-under-18-years}
\end{enumerate}

\subsection{Dealing with Membership Applications}\label{dealing-with-membership-applications}

\begin{enumerate}
  \item The committee must consider each application for membership of the Association and decide whether to accept or reject the application.
  \item The committee may delay its consideration of an application if the committee considers that any matter relating to the application needs to be clarified by the applicant or that the applicant needs to provide further information in support of the application.
  \item The committee must not accept an application unless the applicant --- \label{membership-must-not-be-accepted}
        \begin{enumerate}
          \item is eligible under rule \ref{member-is-eligibile}; and
          \item has applied under rule \ref{member-must-submit-application}; and.
          \item is eligible for the class of membership they have applied for under subrules \ref{eligible-for-class-of-membership}, \ref{eligible-for-voting-rights}, and \ref{members-under-18-years}.
        \end{enumerate}

  \item \label{reasons-why-committee-can-reject-membership-application} The committee may only reject applications for the following reasons ---
        \begin{enumerate}
          \item the committee cannot accept the application because of subrule \ref{membership-must-not-be-accepted}; or
          \item the committee determines that the Association's overall membership capacity has been reached; or
          \item the committee determines that the Association's membership capacity for a particular class of membership has been reached; or
          \item the applicant has previously been expelled from the Association under rule \ref{suspension-or-expulsion}; or
          \item the applicant has previously ceased to be a member under subrule \ref{member-fails-to-pay-fee}; or
          \item \label{reject-membership-detrimental} the applicant has acted detrimentally to the interests of the Association; or
          \item \label{reject-membership-body-corporate} if the applicant is a body corporate - for any reason.
        \end{enumerate}

  \item The committee must notify the applicant of the committee's decision to accept or reject an application as soon as practicable after making the decision.

  \item If the committee rejects the application, the committee is not required to give the applicant its reasons for doing so.
\end{enumerate}



\subsection{Becoming a member} \label{becoming-a-member}


An applicant for membership of the Association becomes a member when ---

\begin{enumerate}
  \def\labelenumi{\alph{enumi})}
  \setcounter{enumi}{0}
  \item the committee accepts the application; and
  \item the applicant pays any membership fees payable to the Association under rule \ref{membership-fees}.
\end{enumerate}



\subsection{Classes of membership} \label{classes-of-membership}

\begin{enumerate}
  \item The Committee may make multiple classes of ordinary membership and associate membership and may make individual arrangements for membership.
  \item A person can only hold one class of membership.
  \item An ordinary member has full voting rights and any other rights conferred on members by these rules or approved by resolution at a general meeting or determined by the committee. \label{ordinary-member-rights}
  \item An associate member has the rights referred to in subrule \ref{ordinary-member-rights} --- \label{associate-member-rights}
        \begin{enumerate}
          \item other than full voting rights;
          \item and rights restricted under a particular class of associate membership.
        \end{enumerate}
  \item The Committee may limit the number of members of any class of membership and associate membership.
  \item \label{swap-ordinary-associate-membership} If a particular class of membership is in use, the Association does not have the right except at a general meeting to change a class of ordinary membership to an class of associate membership or vice versa.
\end{enumerate}


\subsection{When Membership Ceases} \label{when-membership-ceases}

\begin{enumerate}

  \item A person ceases to be a member when any of the following takes place ---

        \begin{enumerate}

          \item for a member who is an individual, the individual dies;
          \item for a member that is a body corporate, the body corporate is wound up;
          \item the person resigns from the Association under rule \ref{resignation};
          \item the person is expelled from the Association under rule \ref{suspension-or-expulsion};
          \item the person ceases to be a member under rule \ref{member-fails-to-pay-fee}.
        \end{enumerate}
  \item The Association must keep a record, for at least one year after a person ceases to be a member, of ---

        \begin{enumerate}

          \item the date on which the person ceased to be a member; and
          \item the reason why the person ceased to be a member.
        \end{enumerate}
\end{enumerate}


\subsection{Resignation} \label{resignation}

\begin{enumerate}

  \item A member may resign from membership of the Association by giving written notice of the resignation to the Secretary or other methods described in the by-laws.
  \item The resignation takes effect ---
        \begin{enumerate}
          \item when the Secretary receives the notice; or
          \item if a later time is stated in the notice, at that later time; or
          \item when described in the by-laws.
        \end{enumerate}
  \item A person who has resigned from membership of the Association remains liable for any fees that are owed to the Association (the \textbf{owed amount}) at the time of resignation.
  \item The owed amount may be recovered by the Association in a court of competent jurisdiction as a debt due to the Association.
\end{enumerate}


\subsection{Rights not Transferable} \label{rights-not-transferable}

The rights of a member are not transferable and end when membership ceases.


\section{Membership fees} \label{division-2-membership-fees}

\subsection{Membership fees} \label{membership-fees}

\begin{enumerate}


  \item \label{fees-set-by} The entrance fee (if any) and the membership fee (if any) for membership of the Association must be ---
        \begin{enumerate}
          \item determined by the committee; and
          \item approved at a general meeting.
        \end{enumerate}

  \item \label{fees-periods} The committee must determine the acceptable schedules for payment of membership fees.

        \note{e.g. monthly, yearly, lifetime.}

  \item The fees determined under subrules \ref{fees-set-by} and \ref{fees-periods} may be different for different classes of membership.

  \item A member must pay the membership fee to the treasurer, or another person authorised by the committee to accept payments, by the date (the \textbf{due date}) determined by the committee. \label{member-must-pay-fee}
  \item If a member has not paid the membership fee within the period of 3 months after the due date, the member ceases to be a member on the expiry of that period. \label{member-fails-to-pay-fee}
  \item \label{member-rejoins-after-failing-to-pay-fee} If a person who has ceased to be a member under subrule \ref{member-fails-to-pay-fee} offers to pay the membership fee for the relevant class of membership after the period referred to in that subrule has expired ---

        \begin{enumerate}

          \item the committee may, at its discretion, accept that payment; and
          \item if the payment is accepted, the person's membership is reinstated from the date the payment is accepted.
        \end{enumerate}
\end{enumerate}

\hypertarget{division-3-register-of-members}{%
  \section{Register of members}\label{division-3-register-of-members}}

\hypertarget{register-of-members}{%
  \subsection{Register of members}\label{register-of-members}}

\begin{enumerate}

  \item The secretary, or another person authorised by the committee, is responsible for the requirements imposed on the Association under section 53 of the Act to maintain the register of members and record in that register any change in the membership of the Association.
  \item In addition to the matters referred to in section 53(2) of the Act, the register of members must include the class of membership (if applicable) to which each member belongs and the date on which each member becomes a member.
  \item The register of members must be kept at a place determined by the committee.
  \item A member who wishes to inspect the register of members must contact the secretary to make the necessary arrangements.
  \item If ---

        \begin{enumerate}
          \item a member inspecting the register of members wishes to make a copy of, or take an extract from, the register under section 54(2) of the Act; or
          \item a member makes a written request under section 56(1) of the Act to be provided with a copy of the register of members,

        \end{enumerate}
        the committee may require the member to provide a statutory declaration setting out the purpose for which the copy or extract is required and declaring that the purpose is connected with the affairs of the Association.
\end{enumerate}

\end{document}