\documentclass[../constitution.tex]{subfiles}
\begin{document}

\part{MEMBERS} \label{part-3-members}


\section{Membership} \label{division-1-membership}


\subsection{Applying for Membership} \label{applying-for-membership}

\begin{enumerate}
\item Any person who is at least 18 years of age and supports the objects or purposes of the Association is eligible to apply to become a member. \label{member-is-eligibile}
\item A person who wishes to become a member of the Association must submit an application to the Committee. \label{member-must-submit-application}
\item The applicant must specify in the application the class of membership.
\end{enumerate}


\subsection{Dealing with Membership Applications}\label{dealing-with-membership-applications}

\begin{enumerate}
\item The committee must consider each application for membership of the Association and decide whether to accept or reject the application.
\item The committee may delay its consideration of an application if the committee considers that any matter relating to the application needs to be clarified by the applicant or that the applicant needs to provide further information in support of the application.
\item The committee must not accept an application unless the applicant ---
  \begin{enumerate}
  \item is eligible under rule \ref{member-is-eligibile}; and
  \item has applied under rule \ref{member-must-submit-application}.
  \end{enumerate}
\item The committee must notify the applicant of the committee's decision to reject an application as soon as practicable after making the decision.
\item If the committee rejects the application, the committee is not required to give the applicant its reasons for doing so.
\end{enumerate}


\subsection{Becoming a member} \label{becoming-a-member}

\begin{enumerate}
\item An applicant for membership of the Association becomes a member when -
  \begin{enumerate}
  \item the applicant pays any membership fees payable to the Association under rule \ref{member-must-pay-fee}; and
  \item the committee accepts the application at a committee meeting or as stated in the by-laws.
  \end{enumerate}
\end{enumerate}


\subsection{Classes of membership} \label{classes-of-membership}

\begin{enumerate}
\item The Committee may make multiple classes of ordinary membership and associate membership and may make individual arrangements for membership.
\item A person or entity can only hold one class of membership.
\item An ordinary member has full voting rights and any other rights conferred on members by these rules or approved by resolution at a general meeting or determined by the committee. \label{ordinary-member-rights}
\item An associate member has the rights referred to in rule \ref{ordinary-member-rights} ---
  \begin{enumerate}
  \item other than full voting rights;
  \item and rights restricted under a particular class of associate membership.
  \end{enumerate}
\item The Committee may limit the number of members of any class of membership and associate membership.
\item If a particular class of membership is in use, the Association does not have the right except at a General Meeting to change a class of membership to an Associate Class or vice versa.
\end{enumerate}


\subsection{When Membership Ceases} \label{when-membership-ceases}

\begin{enumerate}

\item A person ceases to be a member when any of the following takes place ---

  \begin{enumerate}
  
  \item for a member who is an individual, the individual dies;
  \item the person resigns from the Association under rule \ref{resignation};
  \item the person is expelled from the Association under rule \ref{suspension-or-expulsion};
  \item the person ceases to be a member under rule \ref{member-fails-to-pay-fee}
  \end{enumerate}
\item The Association must keep a record, for at least one year after a person ceases to be a member, of ---

  \begin{enumerate}
  
  \item the date on which the person ceased to be a member; and
  \item the reason why the person ceased to be a member.
  \end{enumerate}
\end{enumerate}


\subsection{Resignation} \label{resignation}

\begin{enumerate}

\item A member may resign from membership of the Association by giving written notice of the resignation to the Secretary or other methods described in the by-laws.
\item The resignation takes effect ---
  \begin{enumerate}
  \item when the Secretary receives the notice; or
  \item if a later time is stated in the notice, at that later time; or
  \item when described in the by-laws.
  \end{enumerate}
\item A person who has resigned from membership of the Association remains liable for any fees that are owed to the Association (the owed amount) at the time of resignation.
\item The owed amount may be recovered by the Association in a court of competent jurisdiction as a debt due to the Association.
\end{enumerate}


\subsection{Rights not Transferable} \label{rights-not-transferable}

The rights of a member are not transferable and end when membership ceases.


\section{Membership fees} \label{division-2-membership-fees}

\subsection{Membership fees} \label{membership-fees}

\begin{enumerate}

\item The committee must determine the entrance fee (if any) and the period of the membership fee (if any) to be paid for membership of the Association.
\item The membership fee and payment schedule will be fixed by the Management Committee, subject to review by the members at a general meeting.
\item All members must pay the membership fees on the schedule set by the Management Committee.
\item A member must pay the membership fee to the treasurer, or another person authorised by the committee to accept payments, by the date (the due date) determined by the committee. \label{member-must-pay-fee}
\item If a member has not paid the membership fee within the period of 3 months after the due date, the member ceases to be a member on the expiry of that period. \label{member-fails-to-pay-fee}
\item If a person who has ceased to be a member under rule \ref{member-fails-to-pay-fee} offers to pay the appropriate membership fee relevant to their membership level after the period referred to in that rule has expired ---

  \begin{enumerate}
  
  \item the committee may, at its discretion, accept that payment; and
  \item if the payment is accepted, the person's membership is reinstated from the date the payment is accepted.
  \end{enumerate}
\end{enumerate}

\hypertarget{division-3-register-of-members}{%
\section{Register of members}\label{division-3-register-of-members}}

\hypertarget{register-of-members}{%
\subsection{Register of members}\label{register-of-members}}

\begin{enumerate}

\item The secretary, or another person authorised by the committee, is responsible for the requirements imposed on the Association under section 53 of the Act to maintain the register of members and record in that register any change in the membership of the Association.
\item In addition to the matters referred to in section 53(2) of the Act, the register of members must include the class of membership to which each member belongs and the date on which each member becomes a member.
\item The register of members must be kept at place determined by the committee.
\item A member who wishes to inspect the register of members must contact the secretary to make the necessary arrangements.
\item If ---

  \begin{enumerate}
  \item a member inspecting the register of members wishes to make a copy of, or take an extract from, the register under section 54(2) of the Act; or
  \item a member makes a written request under section 56(1) of the Act to be provided with a copy of the register of members,

    the committee may require the member to provide a statutory declaration setting out the purpose for which the copy or extract is required and declaring that the purpose is connected with the affairs of the Association.
  \end{enumerate}
\end{enumerate}

\end{document}