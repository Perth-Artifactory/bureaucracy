\documentclass[../constitution.tex]{subfiles}
\begin{document}

\part{ASSOCIATION TO BE NOT FOR PROFIT BODY} \label{part-2-association-to-be-not-for-profit-body}


\subsection{Not for Profit Body} \label{not-for-profit-body} 

\begin{enumerate}

\item The property and income of the Association must be applied solely towards the promotion of the objects or purposes of the Association and no part of that property or income may be paid or otherwise distributed, directly or indirectly, to any member, except in good faith in the promotion of those objects or purposes.
\item A payment may be made to a member out of the funds of the Association only if it is authorised under \chreplaced[id=proofing]{subrule}{rule} \ref{reasons-why-members-can-be-paid}
\item A payment to a member out of the funds of the Association is authorised if it is ---

  \begin{enumerate} \label{reasons-why-members-can-be-paid}
  \item the payment in good faith to the member as reasonable remuneration for any services provided to the Association, or for goods supplied to the Association, in the ordinary course of business; or
  \item the payment of interest, on money borrowed by the Association from the member, at a rate not greater than the cash rate published from time to time by the Reserve Bank of Australia; or
  \item the payment of reasonable rent to the member for premises leased by the member to the Association; or
  \item the reimbursement of reasonable expenses properly incurred by the member on behalf of the Association.
  \end{enumerate}
\end{enumerate}

\chcomment[id=proofing]{Informational note added.}

Note for this rule: 

Section 5(1) of the Act provides that an association is not eligible to be 
incorporated under the Act if it is formed or carried on for the purpose 
of securing pecuniary profit for its members from its transactions, and 
section 5(3) of the Act provides details about when an association is 
not ineligible under section 5(1) of the Act. 

\end{document}