\documentclass[../constitution.tex]{subfiles}
\begin{document}

\part{DISCIPLINARY ACTION, DISPUTES AND MEDIATION} \label{part-4-disciplinary-action-disputes-and-mediation}


\section{Term used} \label{division-1-term-used}
% \chcomment{Dvision numbering is broken. This should be div 1. Fixed as of 02/12/22. }

\hypertarget{term-used-member}{%
\subsection{Term Used: Member}\label{term-used-member}}

In this Part \chdeleted[id=proofing]{(Pt \ref{part-4-disciplinary-action-disputes-and-mediation})} ---

\chreplaced[id=proofing]{\textbf{member},}{mmber} in relation to a member who is expelled from the Association, includes former member.

\hypertarget{division-2-disciplinary-action}{%
\section{Disciplinary action}\label{division-2-disciplinary-action}}


Emergency disciplinary actions

15.1 The committee may impose emergency disciplinary actions relating to a member, where ---

* The member has acted detrimentally to the interests of the Association; and
* The committee decides that the member poses an immediate and continuing risk to the interests of the Association, such that a suspension or expulsion under s16 cannot be implemented quickly enough to mitigate the risk.

15.2 Emergency disciplinary actions must be initiated by a committee member and seconded by another committee member.

15.3 Emergency disciplinary actions against a member take effect as soon as a committee member notifies the member, either ---

* verbally; or
* by written notice; or
* by direct message using an electronic communications platform administered by the Association.

15.4 Within 72 hours after notice is given under subrule 15.3, a committee member must deliver a written notice, substantially in the form of Annex A, to ---

* The member
* All committee members

15.5 The duration of an emergency disciplinary action must not exceed 31 days from the date that notice was given under subrule 15.3.

15.5 The member may submit written representations of reasonable length to be considered at the review. Written representations must be recieved by the secretary within 7 days of the date of initial notification.

15.6 Each emergency disciplinary action must be reviewed by the committee within 14 days from the date that notice was given under subrule 15.3. 

The committee must give due consideration to any written representations submitted under subrule 15.5.

The committee must decide whether to ---

* Uphold the emergency disciplinary action; or
* Modify the emergency disciplinary action; or
* Withdraw the emergency disciplinary action.

The committee must give the member written notice of the decision within 3 days.

15.7 The effects of an emergency disciplinary actions may include:

* A restriction or prohibition against using or entering part of, or any of, the Association's premises.
* A restriction or prohibition against using part of, or any of, the Association's equipment, resources, facilities, or services.
* A restriction or prohibition against attending or participating in some of, or any of, events organised by the Association, whether on the Association's premises or not.
* Any other restriction or prohibition which the committee decides is reasonable and appropriate to mitigate a risk to the interests of the Association.

15.8 An emergency disciplinary action cannot have the effect of:

* Preventing a member from starting the grievance procedure (Div 3) or requesting the appointment of a mediator (Div 4).
* Preventing an ordinary member from voting at a general meeting.
* Preventing a committee member from voting at a committee meeting.

15.9 During the period a member is subject to an emergency disciplinary action, the member is not entitled to a refund, rebate, relief, or credit for membership fees paid, or payable, to the Association.




Annex A
-------

Written notice of emergency disciplinary action

This notice relates to an emergency disciplinary action that has been raised according to rule 15 of the rules of the Association.

Name of member subject to the emergency disciplinary action: [XXXXX]

Effects of emergency disciplinary action:

For the duration of this emergency disciplinary action, you must not ---

[XXXXXXXXX]

Your rights:

This decision will be reviewed by the committee within 14 days of the date of initial notification.

You may submit a written representation of reasonable length to be considered at the review. Written representations must be recieved by the secretary within 7 days of the date of initial notification.

Your rights to dispute this emergency disciplinary action are given under Part 4, Division 3 of the rules of the Association.

Your rights to request mediation relating to this emergency disciplinary action are given under Part 4, Division 4 of the rules of the Association.

A copy of the rules of the Association is attached to this notice, or otherwise available at [LINK].


Duration of this emergency disciplinary action

This emergency disciplinary action is effective from the date of initial notice to the expiry date below.

Date of initial notification under rule 15.3:

Expiry date:

Raised by committee member:

Seconded by committee member:



\hypertarget{emergency-temporary-suspension}{%
\subsection{Emergency Temporary Suspension}\label{emergency-temporary-suspension}}

\chcomment{Variance to model rules - no such section exists in the model rules. This section needs cleaning up.}

\begin{enumerate}

\item In the event that the actions of a member pose an imminent risk to the Objects of the Association as defined by this Constitution any two members of the Management Committee may temporarily modify member rights.
\item An emergency temporary suspension cannot be used to

  \begin{enumerate}
  
  \item modify the rights of a member in a way that prevents them from exercising their voting rights as defined by this Constitution.
  \item modify the rights of a member of the Management Committee in a way that prevents them from exercising their ability to vote as a member of the Management Committee.
  \end{enumerate}
\item \textbf{Notice of Emergency Temporary Suspension} An initial notice must be provided verbally or in writing by an instigating Management Committee member. A written notice must be delivered to the suspended member and all members of the committee within 3 days and include -

  \begin{enumerate}
  
  \item The rights being restricted\\
  \item The reason for the Emergency Temporary Suspension\\
  \item A summary of this section of the Constitution\\
  \item Any applicable rights or methods of appeal available to the member
  \item A link to, or a copy of, this Constitution as a whole\\
  \end{enumerate}
\item \textbf{Length of Emergency Temporary Suspension} An Emergency Temporary Suspension under this section must not exceed 31 days from the date of initial notification.\\
\item Upholding and withdrawing an emergency temporary suspension

  \begin{enumerate}
  
  \item An emergency temporary suspension must be reviewed and can be modified, withdrawn or upheld at every subsequent meeting of the management committee where allowable under this constitution.
  \item Any two members of the Management Committee may choose to withdraw or modified an emergency temporary suspension unless that emergency temporary suspension has been upheld by a meeting of the Management Committee. Any changes can be delivered verbally but must also be sent via written notice within 3 days.\\
  \end{enumerate}
\item Appealing an Emergency Temporary Suspension

  \begin{enumerate}
  
  \item A member subject to a emergency temporary suspension defined by this section may opt to provide a written appeal of reasonable length to the Secretary. The Management Committee must consider this appeal when considering the suspension.
  \end{enumerate}
\end{enumerate}

\hypertarget{suspension-or-expulsion}{%
\subsection{Suspension or expulsion}\label{suspension-or-expulsion}}

\begin{enumerate}

\item The committee may decide to suspend a member's membership or to expel a member from the Association if ---

  \begin{enumerate}
  
  \item the member contravenes any of these rules; or
  \item the member acts detrimentally to the interests of the Association.
  \end{enumerate}
\item The secretary must give the member written notice of the \chreplaced[id=proofing]{proposed suspension or expulsion}{suspension or proposed expulsion} at least 28 days before the committee meeting at which the proposal is to be considered by the committee.
\item The notice given to the member must state ---

  \begin{enumerate}
  
  \item when and where the committee meeting is to be held; and
  \item the grounds on which the proposed suspension or expulsion is based; and
  \item that the member, or the member's representative, may attend the meeting and will be given a reasonable opportunity to make written or oral (or both written and oral) submissions to the committee about the proposed suspension or expulsion;
  \end{enumerate}
\item At the committee meeting, the committee must ---

  \begin{enumerate}
  
  \item give the member, or the member's representative, a reasonable opportunity to make written or oral (or both written and oral) submissions to the committee about the proposed suspension or expulsion; and
  \item give due consideration to any submissions so made; and
  \item decide ---

    \begin{enumerate}
    
    \item whether or not to suspend the member's membership and, if the decision is to suspend the membership, the period of suspension; or
    \item whether or not to expel the member from the Association.
    \end{enumerate}
  \end{enumerate}
\item A decision of the committee to suspend the member's membership or to expel the member from the Association takes immediate effect.
\item The committee must give the member written notice of the committee's decision, and the reasons for the decision, within 7 days after the committee meeting at which the decision is made. \label{expulsion-written-notice}
\item A member whose membership is suspended or who is expelled from the Association may, within 14 days after receiving notice of the Committee's decision under \chreplaced[id=proofing]{subrule}{rule} \ref{expulsion-written-notice}, give written notice to the secretary requesting the appointment of a mediator under rule \ref{appointment-of-mediator}. \label{expulsion-appoint-mediator}
\item If notice is given under \chreplaced[id=proofing]{subrule}{rule} \ref{expulsion-appoint-mediator}, the member who gives the notice and the committee are the parties to the mediation.
\end{enumerate}

\hypertarget{consequences-of-suspension}{%
\subsection{Consequences of suspension}\label{consequences-of-suspension}}

\begin{enumerate}

\item During the period a member's membership is suspended, the member ---

  \begin{enumerate}
  
  \item loses any rights (including voting rights) arising as a result of membership; and
  \item is not entitled to a refund, rebate, relief or credit for membership fees paid, or payable, to the Association.
  \end{enumerate}
\item When a member's membership is suspended, the secretary must record in the register of members ---

  \begin{enumerate}
  
  \item that the member's membership is suspended; and
  \item the date on which the suspension takes effect; and
  \item the period of the suspension.
  \end{enumerate}
\item When the period of the suspension ends, the secretary must record in the register of members that the member's membership is no longer suspended.
\end{enumerate}

\hypertarget{division-3-resolving-disputes}{%
\section{Resolving disputes}\label{division-3-resolving-disputes}}

\hypertarget{terms-used}{%
\subsection{Terms Used}\label{terms-used}}

In this Division ---

\chcomment{Some formatting help needed here.}

\textbf{grievance procedure} means the procedures set out in this Division;

\textbf{party to a dispute} includes a person ---

\begin{enumerate}
  \def\labelenumi{\alph{enumi})}
  \setcounter{enumi}{0}
  \item who is a party to the dispute;
  \item and who ceases to be a member within 6 months before the dispute has come to the attention of each party to the dispute.
\end{enumerate}


\hypertarget{application-of-division}{%
\subsection{Application of Division}\label{application-of-division}}

\chreplaced[id=proofing]{The grievance procedure applies to disputes ---}{The procedure set out in this Division (Pt \ref{part-4-disciplinary-action-disputes-and-mediation}, Div \ref{division-3-resolving-disputes}) (the grievance procedure) applies to disputes ---}

\begin{enumerate}
  \def\labelenumi{\alph{enumi})}
  \setcounter{enumi}{0}
  \item between members; or
  \item between one or more members and the Association.
\end{enumerate}

\hypertarget{parties-to-attempt-to-resolve-dispute}{%
\subsection{Parties to attempt to resolve dispute}\label{parties-to-attempt-to-resolve-dispute}}

The parties to a dispute must attempt to resolve the dispute between themselves within 14 days after the dispute has come to the attention of each party.

\hypertarget{how-grievance-procedure-is-started}{%
\subsection{How grievance procedure is started}\label{how-grievance-procedure-is-started}}

\begin{enumerate}

\item If the parties to a dispute are unable to resolve the dispute between themselves within the time required by rule \ref{parties-to-attempt-to-resolve-dispute}, any party to the dispute may start the grievance procedure by giving written notice to the secretary of ---

  \begin{enumerate}
  
  \item the parties to the dispute; and
  \item the matters that are the subject of the dispute.
  \end{enumerate}
\item Within 28 days after the secretary is given the notice, a committee meeting must be convened to consider and determine the dispute.
\item The secretary must give each party to the dispute written notice of the committee meeting at which the dispute is to be considered and determined at least 7 days before the meeting is held.
\item The notice given to each party to the dispute must state ---

  \begin{enumerate}
  
  \item when and where the committee meeting is to be held; and
  \item that the party, or the party's representative, may attend the meeting and will be given a reasonable opportunity to make written or oral (or both written and oral) submissions to the committee about the dispute.
  \end{enumerate}
\item If ---

  \begin{enumerate}
  
  \item the dispute is between one or more members and the Association; and
  \item any party to the dispute gives written notice to the secretary stating that the party ---

    \begin{enumerate}
    
    \item does not agree to the dispute being determined by the committee; and
    \item requests the appointment of a mediator under rule \ref{appointment-of-mediator}, \label{grievance-appoint-mediator}
    \end{enumerate}
  \end{enumerate}
  the committee must not determine the dispute.
\end{enumerate}

\hypertarget{determination-of-dispute-by-committee}{%
\subsection{Determination of dispute by committee}\label{determination-of-dispute-by-committee}}

\begin{enumerate}

\item At the committee meeting at which a dispute is to be considered and determined, the committee must ---

  \begin{enumerate}
  
  \item give each party to the dispute, or the party's representative, a reasonable opportunity to make written or oral (or both written and oral) submissions to the committee about the dispute; and
  \item give due consideration to any submissions so made; and
  \item determine the dispute. \label{committee-determine-dispute}
  \end{enumerate}
\item The committee must give each party to the dispute written notice of the committee's determination, and the reasons for the determination, within 7 days after the committee meeting at which the determination is made.
\item A party to the dispute may, within 14 days after receiving notice of the committee's determination under \chreplaced[id=proofing]{subrule}{rule} \chreplaced[id=proofing]{\ref{committee-determine-dispute}}{22.1.b}, give written notice to the secretary requesting the appointment of a mediator under rule \ref{appointment-of-mediator}. \label{dispute-appoint-mediator}
\item If notice is given under \chreplaced[id=proofing]{subrule}{rule} \ref{dispute-appoint-mediator}, each party to the dispute is a party to the mediation.
\end{enumerate}

\hypertarget{division-4-mediation}{%
\section{Mediation}\label{division-4-mediation}}

\hypertarget{application-of-division-mediation}{%
\subsection{\chreplaced[id=proofing]{Application of division}{Appointment of mediator}}\label{application-of-division-mediation}}

\begin{enumerate}

\item This Division \chdeleted[id=proofing]{(Pt \ref{part-4-disciplinary-action-disputes-and-mediation}, Div \ref{division-4-mediation})} applies if written notice has been given to the secretary requesting the appointment of a mediator ---

  \begin{enumerate}
  
  \item by a member under rule \chreplaced[id=proofing]{\ref{expulsion-appoint-mediator}}{16.6}; or
  \item by a party to a dispute under rule \chreplaced[id=proofing]{\ref{grievance-appoint-mediator}}{21.5.b.i} or \ref{dispute-appoint-mediator}.
  \end{enumerate}
\item If this Division \chdeleted[id=proofing]{(Pt \ref{part-4-disciplinary-action-disputes-and-mediation}, Div \ref{division-4-mediation})} applies, a mediator must be chosen or appointed under rule \ref{appointment-of-mediator}.
\end{enumerate}

\hypertarget{appointment-of-mediator-1}{%
\subsection{Appointment of mediator}\label{appointment-of-mediator}}

\begin{enumerate}

\item \chdeleted[id=proofing, comment={Duplicate text}]{Rule 22- applies if written notice has been given to the secretary requesting the appointment of a mediator ---}

  \begin{enumerate}
  
  \item \chdeleted[id=proofing, comment={Duplicate text}]{by a member under rule \ref{expulsion-written-notice}; or}
  \item \chdeleted[id=proofing, comment={Duplicate text}]{by a party to a dispute under rule \chreplaced[id=proofing]{\ref{grievance-appoint-mediator}}{21.5.b.i} or \ref{dispute-appoint-mediator}.}
  \end{enumerate}
\item The mediator must be a person chosen ---

  \begin{enumerate}
  
  \item if the appointment of a mediator was requested by a member under rule \chreplaced[id=proofing]{\ref{expulsion-appoint-mediator}}{\ref{expulsion-written-notice}} --- by agreement between the Member and the committee; or \label{mediator-chosen-1}
  \item if the appointment of a mediator was requested by a party to a dispute under rule \chreplaced[id=proofing]{\ref{grievance-appoint-mediator}}{21.5.b.i} or \ref{dispute-appoint-mediator} --- by agreement between the parties to the dispute. \label{mediator-chosen-2}
  \end{enumerate}
\item If there is no agreement for the purposes of \chreplaced[id=proofing]{subrule}{rule} \ref{mediator-chosen-1} or \ref{mediator-chosen-2}, then, subject to \chreplaced[id=proofing]{subrules}{rules} \ref{mediator-chosen-3} and \ref{mediator-chosen-4} the committee must appoint the mediator.
\item The person appointed as mediator by the committee must be a person who acts as a mediator for another not-for-profit body, such as a community legal centre, if the appointment of a mediator was requested by --- \label{mediator-chosen-3}

  \begin{enumerate}
  
  \item a member under rule \chreplaced[id=proofing]{\ref{expulsion-appoint-mediator}}{\ref{expulsion-written-notice}}; or
  \item a party to a dispute under rule \chreplaced[id=proofing]{\ref{grievance-appoint-mediator}}{21.5.b.i}; or
  \item a party to a dispute under rule \ref{dispute-appoint-mediator} and the dispute is between one or more members and the Association.
  \end{enumerate}
\item The person appointed as mediator by the committee may be a member or former member of the Association but must not --- \label{mediator-chosen-4}

  \begin{enumerate}
  
  \item have a personal interest in the matter that is the subject of the mediation; or
  \item be biased in favour of or against any party to the mediation.
  \end{enumerate}
\end{enumerate}

\hypertarget{mediation-process}{%
\subsection{Mediation process}\label{mediation-process}}

\begin{enumerate}

\item The parties to the mediation must attempt in good faith to settle the matter that is the subject of the mediation.
\item Each party to the mediation must give the mediator a written statement of the issues that need to be considered at the mediation at least 5 days before the mediation takes place.
\item In conducting the mediation, the mediator must ---

  \begin{enumerate}
  
  \item give each party to the mediation every opportunity to be heard; and
  \item allow each party to the mediation to give due consideration to any written statement given by another party; and
  \item ensure that natural justice is given to the parties to the mediation throughout the mediation process.
  \end{enumerate}
\item The mediator cannot determine the matter that is the subject of the mediation.
\item The mediation must be confidential, and any information given at the mediation cannot be used in any other proceedings that take place in relation to the matter that is the subject of the mediation.
\item The costs of the mediation are to be paid by the party or parties to the mediation that requested the appointment of the mediator.
\end{enumerate}

\note{

Note for this rule: 

Section 182(1) of the Act provides that an application may be made to the State Administrative Tribunal to have a dispute determined if the dispute has not been resolved under the procedure provided for in the incorporated association's rules. 

}

\hypertarget{if-mediation-results-in-decision-to-suspend-or-expel-being-revoked}{%
\subsection{If mediation results in decision to suspend or expel being revoked}\label{if-mediation-results-in-decision-to-suspend-or-expel-being-revoked}}

If ---

  \begin{enumerate}
  \def\labelenumi{\alph{enumi})}
  \setcounter{enumi}{0}
  \item mediation takes place because a member whose membership is suspended or who is expelled from the Association gives notice under rule \chreplaced[id=proofing]{\ref{expulsion-appoint-mediator}}{\ref{expulsion-written-notice}}; and
  \item as the result of the mediation, the decision to suspend the member's membership or expel the member is revoked,
  \end{enumerate}

that revocation does not affect the validity of any decision made at a committee meeting or general meeting during the period of suspension or expulsion.


\end{document}