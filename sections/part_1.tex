\documentclass[../constitution.tex]{subfiles}
\begin{document}

\part{PRELIMINARY} \label{part-1---preliminary}

\subsection{Definitions} \label{definitions}

In these rules, unless the contrary intention appears ---

\begin{itemize}[label={-}]
\item \textbf{Act} means the \textit{Associations Incorporation Act 2015};
\item \textbf{associate member} means a member with the rights referred to in rule \ref{associate-member-rights};
\item \textbf{Association} means the incorporated association to which these rules apply;
\item \textbf{books} of the Association, includes the following ---

  \begin{enumerate}
  \def\labelenumi{\alph{enumi})}
  \setcounter{enumi}{0}
  
  \item a register;
  \item financial records, financial statements or financial reports, however compiled, recorded or stored;
  \item a document;
  \item any other record of information;
  \end{enumerate}
\item \textbf{by-laws} means by-laws made by the Association under rule \ref{by-laws};
\item \textbf{chairperson\chdeleted[id=proofing]{ / chair}} means the \chreplaced[id=proofing]{committee}{Committee} member holding office as the chairperson of the Association;
\item \textbf{Commissioner} means the person for the time being designated as the Commissioner under section 153 of the Act;
\item \textbf{committee} means the management committee of the Association;
\item \textbf{committee meeting} means a meeting of the committee;
\item \textbf{committee member} means a member of the committee;
\item \textbf{executive committee} means the office \chreplaced[id=proofing]{holders}{bearers} of the Association; \chcomment{"Executive committee" is not used anywhere in this document. Is it used in bylaws? Suggest deleting this term.}
\item \textbf{financial records} includes ---

  \begin{enumerate}
    \def\labelenumi{\alph{enumi})}
    \setcounter{enumi}{0}
  
  \item invoices, receipts, orders for the payment of money, bills of exchange, cheques, promissory notes and vouchers; and
  \item documents of prime entry; and
  \item working papers and other documents needed to explain ---
    \begin{enumerate}
    \def\labelenumi{\roman{enumi})}
    \setcounter{enumi}{0}
    \item the methods by which financial statements are prepared; and
    \item adjustments to be made in preparing financial statements; \chcomment{Need some formatting help here. These should be labelled i) and ii).}
    \end{enumerate}
  \end{enumerate}
\item \textbf{financial report}, of a tier 2 association or a tier 3 association, has the meaning given in section 63 of the Act;
\item \textbf{financial statements} means the financial statements in relation to the Association required under Part 5 Division 3 of the Act;
\item \textbf{financial year}\chadded[id=proofing]{,} of the Association, has the meaning given in rule \ref{financial-year};
\item \textbf{general meeting}\chadded[id=proofing]{,} of the Association, means a meeting of the Association that all members are entitled to receive notice of and to attend;
\item \textbf{member} means a person who is an ordinary member or an associate member of the Association; \chcomment{Variance from model rules - model rules permit body corporates to be members.}
\item \textbf{\chreplaced[id=proofing]{office holder}{officer bearer}} -- refers to the Chairperson, Treasurer, Secretary, Vice Chairperson \chcomment{Variance to model rules: The term "office holder" is defined in \ref{officebearer-roles} rather than being defined here. Suggest deleting this definition, or changing it to: "office holder means the committee members listed in subrule \ref{officebearer-roles}. }
\item \textbf{ordinary committee member} means a committee member who is not an office holder of the Association under \chreplaced[id=proofing]{\ref{officebearer-roles}}{rule 28.1};
\item \textbf{ordinary member} means a member with the rights referred to in rule \ref{ordinary-member-rights};
\item \textbf{register of members} means the register of members referred to in section 53 of the Act;
\item \textbf{rules} means these rules of the Association, as in force for the time being;
\item \textbf{secretary} means the committee member holding office as the secretary of the Association;
\item \textbf{special general meeting (SGM)} means a general meeting of the Association other than the annual general meeting;
\item \textbf{special resolution} means a resolution passed by the members at a general meeting in accordance with section 51 of the Act;
\item \textbf{subcommittee} means a subcommittee appointed by the committee under rule \ref{appoint-subcommittees};
\item \textbf{tier 1 association} means an incorporated association to which section 64(1) of the Act applies;
\item \textbf{tier 2 association} means an incorporated association to which section 64(2) of the Act applies;
\item \textbf{tier 3 association} means an incorporated association to which section 64(3) of the Act applies;
\item \textbf{treasurer} means the committee member holding office as the treasurer of the Association.
\end{itemize}


\subsection{Purpose/Objects}\label{purposeobjects}

\chcomment{This is a duplicate of what is in the "guidance notes". Should be de-duplicated.}

The Purpose of the Association is to encourage and facilitate creative use of technology. The objects of the Association are to:

\begin{enumerate}

\item promote the creative use of technology;
\item establish, maintain, and equip a shared work space for its members;
\item provide work space, storage, and other resources for the creative and artistic use of technology;
\item foster a collaborative, inclusive, safe, and creative environment for artistic and technological projects;
\item educate and train its members in skills relevant to its objects;
\item organise educational, social and cultural events to promote the creative use of technology;
\item raise funds to support its other objects;
\item communicate and collaborate with others with similar objectives.
\end{enumerate}


\subsection{Financial year} \label{financial-year}

The Association's financial year will be the period of 12 months commencing on 1st July and ending on the 30th June each year.

\end{document}