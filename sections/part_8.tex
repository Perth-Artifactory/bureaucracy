\documentclass[../constitution.tex]{subfiles}
\begin{document}

\hypertarget{part-8-general-matters}{%
  \part{GENERAL MATTERS}\label{part-8-general-matters}}

\hypertarget{by-laws}{%
  \subsection{By-laws}\label{by-laws}}

\begin{enumerate}

  \item The committee shall have the power to make, alter and rescind any by-laws that the committee considers necessary for the effective administration of the Association.
  \item The Association may, by resolution at a general meeting, make, amend or revoke by-laws.
  \item By-laws may ---

        \begin{enumerate}

          \item provide for the rights and obligations that apply to any classes of membership approved under rule \ref{classes-of-membership}; and
          \item impose restrictions on the committee's powers, including the power to dispose of the Association's assets; and
          \item impose requirements relating to the financial reporting and financial accountability of the association and the auditing of the Association's accounts; and \label{by-law-may-impose-financial-reporting-requirements}
          \item provide for any other matter the Association considers necessary or convenient to be dealt with in the by-laws.
        \end{enumerate}
  \item A by-law is of no effect to the extent that it is inconsistent with the Act, the regulations or these rules. \label{by-law-no-effect-if-inconsistent-with-act}
  \item Without limiting rule \ref{by-law-no-effect-if-inconsistent-with-act}, a by-law made for the purposes of rule \ref{by-law-may-impose-financial-reporting-requirements} may only impose requirements on the Association that are additional to, and do not restrict, a requirement imposed on the Association under Part 5 of the Act.
  \item At the request of a member, the Association must make a copy of the by-laws available for inspection by the member.
\end{enumerate}

\hypertarget{executing-documents}{%
  \subsection{Executing documents}\label{executing-documents}}

\begin{enumerate}

  \item The Association may execute a document if the document is signed by ---

        \begin{enumerate}

          \item 2 committee members; or
          \item one committee member and a person authorised by the committee.
        \end{enumerate}
  \item The secretary must make a written record of each document executed.
  \item The Association does not have a common seal. \label{no-common-seal}
\end{enumerate}

\hypertarget{giving-notices-to-members}{%
  \subsection{Giving notices to members}\label{giving-notices-to-members}}

\begin{enumerate}

  \item In this rule ---

        \textbf{recorded} means recorded in the register of members.

        \textbf{direct message} has the meaning given in rule \ref{terms-used}.

  \item A notice or other document that is to be given to a member under these rules is taken not to have been given to the member unless it is in writing and ---

        \begin{enumerate}
          \item sent by electronic transmission to an appropriate recorded electronic address of the member; or
          \item delivered by hand to the recorded physical address of the member; or
          \item sent by prepaid post to the recorded postal address of the member; or
          \item sent by direct message to the member, for a type of notice which these rules allow to be given by direct message.
        \end{enumerate}
\end{enumerate}

\hypertarget{custody-of-books-and-securities}{%
  \subsection{Custody of books and securities}\label{custody-of-books-and-securities}}

\begin{enumerate}

  \item Subject to subrule \ref{treasurer-keeps-custody-of-financial-reports}, the books and any securities of the Association must be kept in the secretary's custody or under the secretary's control. \label{secretary-keeps-custody-of-books}
  \item The financial records and, as applicable, the financial statements or financial reports of the Association must be kept in the treasurer's custody or under the treasurer's control. \label{treasurer-keeps-custody-of-financial-reports}
  \item Subrules \ref{secretary-keeps-custody-of-books} and \ref{treasurer-keeps-custody-of-financial-reports} have effect except as otherwise decided by the committee.
  \item The books of the Association must be retained for at least 7 years.
\end{enumerate}

\hypertarget{record-of-office-holders}{%
  \subsection{Record of office holders}\label{record-of-office-holders}}

The record of committee members and other persons authorised to act on behalf of the Association that is required to be maintained under section 58(2) of the Act must be kept in the secretary's custody or under the secretary's control.

\note[Note for this rule]{

  Section 58 of the Act —

  (a) sets out the details of the record that an incorporated association must maintain of the committee members and certain others; and

  (b) provides for members to inspect, make a copy of or take an extract from the record; and

  (c) prohibits a person from disclosing information in the record except for authorised purposes.

}

\hypertarget{inspection-of-records-and-documents}{%
  \subsection{Inspection of records and documents}\label{inspection-of-records-and-documents}}

\begin{enumerate}

  \item Subrule \ref{member-contact-secretary-for-inspection} applies to a member who wants to inspect ---

        \begin{enumerate}

          \item the register of members under section 54(1) of the Act; or
          \item the record of the names and addresses of committee members, and other persons authorised to act on behalf of the Association, under section 58(3) of the Act; or
          \item any other record or document of the association. \label{any-other-record}
        \end{enumerate}
  \item The member must contact the secretary to make the necessary arrangements for the inspection. \label{member-contact-secretary-for-inspection}
  \item The inspection must be free of charge.
  \item If the member wants to inspect a document that records the minutes of a committee meeting, the right to inspect that document is subject to any decision the committee has made about minutes of committee meetings generally, or the minutes of a specific committee meeting, being available for inspection by members.
  \item The member may make a copy of or take an extract from a record or document referred to in subrule \ref{any-other-record} but does not have a right to remove the record or document for that purpose.

        \note[Note for this subrule]{

          Sections 54(2) and 58(4) of the Act provide for the making of copies of, or the taking of extracts from, the register referred to in subrule (1)(a) and the record referred to in subrule (1)(b).

        }

  \item The member must not use or disclose information in a record or document referred to in subrule \ref{any-other-record} except for a purpose ---

        \begin{enumerate}

          \item that is directly connected with the affairs of the Association; or
          \item that is related to complying with a requirement of the Act.
        \end{enumerate}

        \note[Note for this subrule]{

          Sections 57(1) and 58(5) of the Act impose restrictions on the use or  disclosure of information in the register referred to in subrule (1)(a) and  the record referred to in subrule (1)(b).

        }

\end{enumerate}

\hypertarget{publication-by-committee-members-of-statements-about-association-business-prohibited}{%
  \subsection{Publication by committee members of statements about Association business prohibited}\label{publication-by-committee-members-of-statements-about-association-business-prohibited}}

A committee member must not publish, or cause to be published, any statement about the business conducted by the Association at a general meeting or committee meeting unless ---

\begin{enumerate}

  \item the committee member has been authorised to do so at a committee meeting; and
  \item the authority given to the committee member has been recorded in the minutes of the committee meeting at which it was given.
\end{enumerate}


\hypertarget{distribution-of-surplus-property-on-cancellation-of-incorporation-or-winding-up}{%
  \subsection{Distribution of surplus property on cancellation of incorporation or winding up}\label{distribution-of-surplus-property-on-cancellation-of-incorporation-or-winding-up}}

\begin{enumerate}

  \item In this rule -

        \textbf{surplus property}, in relation to the Association, means property remaining after satisfaction of ---

        \begin{enumerate}
          \item the debts and liabilities of the Association; and
          \item the costs, charges and expenses of winding up or cancelling the incorporation of the Association,
        \end{enumerate}
        but does not include books relating to the management of the Association.
  \item On the cancellation of the incorporation or the winding up of the Association, its surplus property must be distributed as determined by special resolution by reference to the persons mentioned in section 24(1) of the Act.

  \item If the Association is wound up or its endorsement as a deductible gift recipient is revoked (whichever occurs first), any surplus of the following assets shall be transferred to another organisation with similar objects, which is charitable at law, to which income tax deductible gifts can be made ---
        \begin{enumerate}
          \item gifts of money or property for the principal purpose of the organisation;
          \item contributions made in relation to an eligible fundraising event held for the principal purpose of the Association
          \item money received by the Association because of such gifts and contributions
        \end{enumerate}
  \item If the Association is wound up, after it has paid all debts and other liabilities (including the costs of winding up), any remaining assets ---
        \begin{enumerate}
          \item must not be distributed to the members or former members of the Association, and
          \item subject to the requirements of Australian laws and any Australian court order, must be distributed to another organisation or other organisations, with similar purposes, which is/are charitable at law, and which is/are not carried on for the profit or personal gain of members.
        \end{enumerate}
\end{enumerate}


\note[Note for this rule]{

  Section 24(1) of the Act sets out a provision that is implied in these rules describing the entities to which the surplus property of an incorporated association may be distributed on the cancellation of the incorporation or the winding up of the association.  Part 9 of the Act deals with the winding up of incorporated associations, and Part 10 of the Act deals with the cancellation of the incorporation of incorporated associations.

}

\hypertarget{alteration-of-rules}{%
  \subsection{Alteration of rules}\label{alteration-of-rules}}

If the Association wants to alter or rescind any of these rules, or to make additional rules, the Association may do so only by special resolution and by otherwise complying with Part 3 Division 2 of the Act.

\note[Note for this rule]{

  Section 31 of the Act requires an incorporated association to obtain the Commissioner's approval if the alteration of its rules has effect to change the name of the association.  Section 33 of the Act requires an incorporated association to obtain the Commissioner's approval if the alteration of its rules has effect to alter the objects or purposes of the association or the manner in which surplus property of the association must be distributed or dealt with if the association is wound up or its incorporation is cancelled.

}

\note {
  Any alteration of the rules must be \href{https://www.commerce.wa.gov.au/books/inc-guide-incorporated-associations-western-australia/step-step-guide}{lodged with the Commissioner} within one month of the passage of the special resolution.
}

\end{document}