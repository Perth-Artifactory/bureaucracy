\documentclass[../constitution.tex]{subfiles}
\begin{document}

\hypertarget{part-5-committee}{%
\part{COMMITTEE}\label{part-5-committee}}

\hypertarget{division-1-powers-of-committee}{%
\section{Powers of Committee}\label{division-1-powers-of-committee}}

\hypertarget{committee}{%
\subsection{Committee}\label{committee}}

\begin{enumerate}

\item The committee members are the persons who, as the management committee of the Association, have the power to manage the affairs of the Association.
\item Subject to the Act, these rules, the by-laws (if any) and any resolution passed at a general meeting, the committee has power to do all things necessary or convenient to be done for the proper management of the affairs of the Association.
\item The committee must take all reasonable steps to ensure that the Association complies with the Act, these rules and the by-laws.
\end{enumerate}

\hypertarget{division-2-composition-of-committee-and-duties-of-members}{%
\section{Composition of Committee and duties of members}\label{division-2-composition-of-committee-and-duties-of-members}}

\hypertarget{committee-members}{%
\subsection{Committee members}\label{committee-members}}

\begin{enumerate}

\item The Management Committee must consist of at least 5 but not more than 10 members.
\item The following are the office bearers of the Association ---

  \begin{enumerate}
  
  \item the chairperson;
  \item the deputy chairperson;
  \item the secretary;
  \item the treasurer.
  \end{enumerate}
\item A person may be a committee member if the person is ---

  \begin{enumerate}
  
  \item an individual who has reached 18 years of age; and
  \item a ordinary member.
  \end{enumerate}
\item A person must not hold 2 or more of the offices mentioned in rule 28.2 at the same time.
\item The office bearers referred to in rule 28.2 will form the Executive Committee
\end{enumerate}

\hypertarget{chairperson}{%
\subsection{Chairperson}\label{chairperson}}

\begin{enumerate}

\item It is the duty of the chairperson to consult with the secretary regarding the business to be conducted at each committee meeting and general meeting.
\item The chairperson has the powers and duties relating to convening and presiding at committee meetings and presiding at general meetings provided for in these rules.
\end{enumerate}

\hypertarget{secretary}{%
\subsection{Secretary}\label{secretary}}

The Secretary has the following duties ---

\begin{enumerate}
  \item dealing with the Association's correspondence;
  \item consulting with the chairperson regarding the business to be conducted at each committee meeting and general meeting;
  \item preparing the notices required for meetings and for the business to be conducted at meetings;
  \item unless another member is authorised by the committee to do so, maintaining on behalf of the Association the register of members, and recording in the register any changes in the membership, as required under section 53(1) of the Act;
  \item maintaining on behalf of the Association an up-to-date copy of these rules, as required under section 35(1) of the Act;
  \item unless another member is authorised by the committee to do so, maintaining on behalf of the Association a record of committee members and other persons authorised to act on behalf of the Association, as required under section 58(2) of the Act;
  \item ensuring the safe custody of the books of the Association, other than the financial records, financial statements and financial reports, as applicable to the Association;
  \item maintaining full and accurate minutes of committee meetings and general meetings;
  \item carrying out any other duty given to the secretary under these rules or by the committee.
\end{enumerate}

\hypertarget{treasurer}{%
\subsection{Treasurer}\label{treasurer}}

The treasurer has the following duties ---

\begin{enumerate}
  \item ensuring that any amounts payable to the Association are collected and issuing receipts for those amounts in the Association's name;
  \item ensuring that any amounts paid to the Association are credited to the appropriate account of the Association, as directed by the committee;
  \item ensuring that any payments to be made by the Association that have been authorised by the committee or at a general meeting are made on time;
  \item ensuring that the Association complies with the relevant requirements of Part 5 of the Act;
  \item ensuring the safe custody of the Association's financial records, financial statements and financial reports, as applicable to the Association;
  \item if the Association is a tier 1 association, coordinating the preparation of the Association's financial statements before their submission to the Association's annual general meeting;
  \item if the Association is a tier 2 association or tier 3 association, coordinating the preparation of the Association's financial report before its submission to the Association's annual general meeting;
\item h providing any assistance required by an auditor or reviewer conducting an audit or review of the Association's financial statements or financial report under Part 5 Division 5 of the Act;
  
  \item carrying out any other duty given to the treasurer under these rules or by the committee.
\end{enumerate}

\hypertarget{division-3-election-of-committee-members-and-tenure-of-office}{%
\section{Election of committee members and tenure of office}\label{division-3-election-of-committee-members-and-tenure-of-office}}

\hypertarget{how-members-become-committee-members}{%
\subsection{How members become Committee members}\label{how-members-become-committee-members}}

A member becomes a committee member if the member ---

\begin{enumerate}

\item is elected to the committee at a general meeting; or
\item is appointed to the committee by the committee to fill a casual vacancy under rule 39.
\end{enumerate}

\hypertarget{nomination-of-committee-members}{%
\subsection{Nomination of committee members}\label{nomination-of-committee-members}}

\begin{enumerate}

\item At least 21 days before an annual general meeting, the secretary must send written notice to all the members ---

  \begin{enumerate}
  
  \item calling for nominations for election to the committee; and
  \item stating the date by which nominations must be received by the secretary to comply with rule 33.2
  \end{enumerate}
\item A member who wishes to be considered for election to the committee at the annual general meeting must nominate for election by sending written notice of the nomination to the secretary at least 7 days before the annual general meeting.
\item The written notice must include a statement by another member in support of the nomination.
\item A member may nominate to be a committee member.
\item A member whose nomination does not comply with this rule is not eligible for election to the committee unless the member is nominated under rule 34.2.
\end{enumerate}

\hypertarget{election-of-ordinary-committee-members}{%
\subsection{Election of ordinary committee members}\label{election-of-ordinary-committee-members}}

\begin{enumerate}

\item At the annual general meeting, the Association must decide by resolution the number of ordinary committee members (if any) to hold office for the next year.
\item If the number of members nominating for the position of ordinary committee member is not greater than the number to be elected, the chairperson of the meeting ---

  \begin{enumerate}
  
  \item must declare each of those members to be elected to the position; and
  \item may call for further nominations from the ordinary members at the meeting to fill any positions remaining unfilled after the elections under 34.2.a
  \end{enumerate}
\item If ---

  \begin{enumerate}
  
  \item the number of members nominating for the position of ordinary committee member is greater than the number to be elected; or
  \item the number of members nominating under rule 34.2.b is greater than the number of positions remaining unfilled,
  \end{enumerate}
\item the members at the meeting must vote by secret ballot to decide the members who are to be elected to the position of ordinary committee member.
\item A member who has nominated for the position of ordinary committee member may vote in accordance with that nomination.
\item Proxy votes will be allowed under rule 54. Proxy votes must reach the Secretary prior to the commencement of the meeting.
\end{enumerate}

\hypertarget{election-of-office-bearers}{%
\subsection{Election of office bearers}\label{election-of-office-bearers}}

\begin{enumerate}

\item The office bearers of the Association are determined by the elected committee after the annual general meeting.
\item The newly elected Committee must decide by vote on office bearers of the Association and any other officers/roles for the Association.
\item If there is no nomination for a position, the chairperson of the meeting will appoint a member.
\item If only one member has nominated for a position, the chairperson of the meeting must declare the Member elected to the position.
\item If more than one committee member has nominated for a position, the committee members at the meeting must vote by secret ballot.
\item Each committee member present at the meeting may vote for one member who has nominated for the position.
\item A committee member who has nominated for the position may vote for themselves.
\item On the committee member's election, the new chairperson of the Association may take over as the chairperson of the meeting.
\end{enumerate}

\hypertarget{term-of-office}{%
\subsection{Term of office}\label{term-of-office}}

\begin{enumerate}

\item The term of office of a committee member begins when the member ---

  \begin{enumerate}
  
  \item is elected at an annual general meeting or under rule 37.3.b; or
  \item is appointed to fill a casual vacancy under rule 39.
  \end{enumerate}
\item Subject to rule 36, a committee member holds office until the positions on the committee are declared vacant at the next annual general meeting.
\item A committee member may be re-elected.
\end{enumerate}

\hypertarget{resignation-and-removal-from-office}{%
\subsection{Resignation and removal from office}\label{resignation-and-removal-from-office}}

\begin{enumerate}

\item A committee member may resign from the committee by written notice given to the secretary or, if the resigning member is the secretary, given to the chairperson.
\item The resignation takes effect ---

  \begin{enumerate}
  
  \item when the notice is received by the secretary or chairperson; or
  \item if a later time is stated in the notice, at the later time.
  \end{enumerate}
\item At a general meeting, the Association may by resolution ---

  \begin{enumerate}
  
  \item remove a committee member from office; and
  \item elect a member who is eligible under rule 28.3 to fill the vacant position.
  \end{enumerate}
\item A committee member who is the subject of a proposed resolution under rule 37.3.a may make written representations (of a reasonable length) to the secretary or chairperson and may ask that the representations be provided to the members.
\item The secretary or chairperson may give a copy of the representations to each member or, if they are not so given, the committee member may require them to be read out at the general meeting at which the resolution is to be considered.
\end{enumerate}

\hypertarget{when-membership-of-committee-ceases}{%
\subsection{When membership of committee ceases}\label{when-membership-of-committee-ceases}}

\begin{enumerate}

\item A person ceases to be a committee member if the person ---

  \begin{enumerate}
  
  \item dies or otherwise ceases to be a member; or
  \item resigns from the committee or is removed from office under rule 37; or
  \item becomes ineligible to accept an appointment or act as a committee member under section 39 of the Act;
  \item becomes permanently unable to act as a committee member because of a mental or physical disability; or
  \item fails to attend 3 consecutive Committee meetings without having notified the Committee at least 24 hours before hand that the person will be unable to attend.
  \end{enumerate}
\end{enumerate}

\hypertarget{filling-casual-vacancies}{%
\subsection{Filling casual vacancies}\label{filling-casual-vacancies}}

\begin{enumerate}

\item The committee may appoint a member who is eligible under rule 28.3 to fill a position on the committee that ---

  \begin{enumerate}
  
  \item has become vacant under rule 38; or
  \item was not filled by election at the most recent annual general meeting or under rule 37.3.b.
  \end{enumerate}
\item If the position of secretary becomes vacant, the committee must appoint a member who is eligible under rule 28.3 to fill the position within 14 days after the vacancy arises.
\item Subject to the requirement for a quorum under rule 46, the committee may continue to act despite any vacancy in its membership.
\item If there are fewer committee members than required for a quorum under rule 46, the committee may act only for the purpose of ---

  \begin{enumerate}
  
  \item appointing committee members under this rule; or
  \item convening a general meeting.
  \end{enumerate}
\end{enumerate}

\hypertarget{validity-of-acts}{%
\subsection{Validity of acts}\label{validity-of-acts}}

The acts of a committee or subcommittee, or of a committee member or member of a subcommittee, are valid despite any defect that may afterwards be discovered in the election, appointment or qualification of a committee member or member of a subcommittee.

\hypertarget{payments-to-committee-members}{%
\subsection{Payments to Committee Members}\label{payments-to-committee-members}}

\begin{enumerate}

\item In this rule --- committee member includes a member of a subcommittee;
\item committee meeting includes a meeting of a subcommittee.
\item A committee member is not entitled to be paid out of the funds of the Association for any out-of-pocket expenses for travel and accommodation incurred ---

  \begin{enumerate}
  
  \item in attending a committee meeting; or
  \item in attending a general meeting; or
  \item otherwise in connection with the Association's business.
  \end{enumerate}
\end{enumerate}

\hypertarget{division-4-committee-meetings}{%
\section{Committee meetings}\label{division-4-committee-meetings}}

\hypertarget{committee-meetings}{%
\subsection{Committee meetings}\label{committee-meetings}}

\begin{enumerate}

\item The Management Committee will meet at least once every two months.
\item The date, time and place of the first committee meeting must be determined by the committee members as soon as practicable after the annual general meeting at which the committee members are elected.
\item Special committee meetings may be convened by the chairperson or any 3 committee members.
\end{enumerate}

\hypertarget{notice-of-committee-meetings}{%
\subsection{Notice of committee meetings}\label{notice-of-committee-meetings}}

\begin{enumerate}

\item Notice of each committee meeting must be given to each committee member at least 48 hours before the time of the meeting.
\item The notice must state the date, time and place of the meeting and must describe the general nature of the business to be conducted at the meeting.
\item Unless rule 43.4 applies, the only business that may be conducted at the meeting is the business described in the notice.
\item Urgent business that has not been described in the notice may be conducted at the meeting if the committee members at the meeting unanimously agree to treat that business as urgent.
\end{enumerate}

\hypertarget{procedure-and-order-of-business}{%
\subsection{Procedure and order of business}\label{procedure-and-order-of-business}}

\begin{enumerate}

\item The chairperson or, in the chairperson's absence, the deputy-chairperson must preside as chairperson of each committee meeting.
\item If the chairperson and deputy chairperson are absent or are unwilling to act as chairperson of a meeting, the committee members at the meeting must choose one of them to act as chairperson of the meeting.
\item The procedure to be followed at a committee meeting must be determined from time to time by the committee.
\item The order of business at a committee meeting may be determined by the committee members at the meeting.
\item Members or other person who is not a committee member may attend a committee meeting if invited to do so by the committee.
\item A person invited under rule 44.5 to attend a committee meeting ---

  \begin{enumerate}
  
  \item has no right to any agenda, minutes or other document circulated at the meeting; and
  \item must not comment about any matter discussed at the meeting unless invited by the committee to do so; and
  \item cannot vote on any matter that is to be decided at the meeting.
  \end{enumerate}
\end{enumerate}

\hypertarget{use-of-technology-to-be-present-at-committee-meetings}{%
\subsection{Use of technology to be present at committee meetings}\label{use-of-technology-to-be-present-at-committee-meetings}}

\begin{enumerate}

\item Participation in meetings via electronic communications technology shall be considered attendance if acceptable to the Secretary and Chair.
\item A member who participates in a committee meeting as allowed under rule (45.1) is taken to be present at the meeting and, if the member votes at the meeting, the member is taken to have voted in person.
\end{enumerate}

\hypertarget{quorum-for-committee-meetings}{%
\subsection{Quorum for committee meetings}\label{quorum-for-committee-meetings}}

\begin{enumerate}

\item Quorum for a Committee Meeting is stated in Guidance Note D
\item Subject to rule 39.4 no business is to be conducted at a committee meeting unless a quorum is present.
\item If a quorum is not present within 30 minutes after the notified commencement time of a committee meeting ---

  \begin{enumerate}
  
  \item the committee members present may elect to wait up to another 60 minutes for the quorum to be present
  \item in the case of a special meeting --- the meeting lapses; or
  \item otherwise, the meeting is adjourned to the same time, day and place in the following week.
  \end{enumerate}
\item If ---

  \begin{enumerate}
  
  \item a quorum is not present within 30 minutes after the commencement time of a committee meeting held under rule 46.3.c; and
  \item at least 2 committee members are present at the meeting,
  \end{enumerate}
\item those members present are taken to constitute a quorum.
\end{enumerate}

\hypertarget{voting-at-committee-meetings}{%
\subsection{Voting at committee meetings}\label{voting-at-committee-meetings}}

\begin{enumerate}

\item Each committee member present at a committee meeting has one vote on any question arising at the meeting.
\item A motion is carried if a majority of the committee members present at the committee meeting vote in favour of the motion.
\item If the votes are divided equally on a question, the chairperson of the meeting has a second or casting vote.
\item A vote may take place by the committee members present indicating their agreement or disagreement or by a show of hands, unless the committee decides that a secret ballot is needed to determine a particular question.
\item If a secret ballot is needed, the chairperson of the meeting must decide how the ballot is to be conducted.
\end{enumerate}

\hypertarget{minutes-of-committee-meetings}{%
\subsection{Minutes of committee meetings}\label{minutes-of-committee-meetings}}

\begin{enumerate}

\item The committee must ensure that minutes are taken and kept of each committee meeting.
\item The minutes must record the following ---

  \begin{enumerate}
  
  \item the names of the committee members present at the meeting;
  \item the name of any person attending the meeting under rule 44.5;
  \item the business considered at the meeting;
  \item any motion on which a vote is taken at the meeting and the result of the vote.
  \end{enumerate}
\item The minutes of a committee meeting must be entered in the Association's minute record within 7 days after the meeting is held.
\item The committee must ensure that the minutes of a committee meeting are reviewed as correct by a motion at the next committee meeting.
\item When the minutes of a committee meeting have been accepted as correct they are, until the contrary is proved, evidence that ---

  \begin{enumerate}
  
  \item the meeting to which the minutes relate was duly convened and held; and
  \item the matters recorded as having taken place at the meeting took place as recorded; and
  \item any appointment purportedly made at the meeting was validly made.
  \end{enumerate}
\end{enumerate}

\hypertarget{division-5-subcommittees-and-subsidiary-offices}{%
\section{Subcommittees and subsidiary offices}\label{division-5-subcommittees-and-subsidiary-offices}}

\hypertarget{subcommittees-and-subsidiary-offices}{%
\subsection{Subcommittees and subsidiary offices}\label{subcommittees-and-subsidiary-offices}}

\begin{enumerate}

\item To help the committee in the conduct of the Association's business, the committee may, in writing, do either or both of the following ---

  \begin{enumerate}
  
  \item appoint one or more subcommittees;
  \item create one or more subsidiary offices and appoint people to those offices.
  \end{enumerate}
\item A subcommittee may consist of the number of people, whether or not members, that the committee considers appropriate.
\item A person may be appointed to a subsidiary office whether or not the person is a member.
\item Subject to any directions given by the committee ---

  \begin{enumerate}
  
  \item a subcommittee may meet and conduct business as it considers appropriate; and
  \item the holder of a subsidiary office may carry out the functions given to the holder as the holder considers appropriate.
  \end{enumerate}
\end{enumerate}

\hypertarget{delegation-to-subcommittees-and-holders-of-subsidiary-offices}{%
\subsection{Delegation to subcommittees and holders of subsidiary offices}\label{delegation-to-subcommittees-and-holders-of-subsidiary-offices}}

\begin{enumerate}

\item In this rule ---
\item \textbf{non-delegable duty} means a duty imposed on the committee by the Act or another written law.
\item The committee may, in writing, delegate to a subcommittee or the holder of a subsidiary office the exercise of any power or the performance of any duty of the committee other than ---

  \begin{enumerate}
  
  \item the power to delegate; and
  \item a non-delegable duty.
  \end{enumerate}
\item A power or duty, the exercise or performance of which has been delegated to a subcommittee or the holder of a subsidiary office under this rule, may be exercised or performed by the subcommittee or holder in accordance with the terms of the delegation.
\item The delegation may be made subject to any conditions, qualifications, limitations or exceptions that the committee specifies in the document by which the delegation is made.
\item The delegation does not prevent the committee from exercising or performing at any time the power or duty delegated.
\item Any act or thing done by a subcommittee or by the holder of a subsidiary office, under the delegation has the same force and effect as if it had been done by the committee.
\item The committee may, in writing, amend or revoke the delegation.
\end{enumerate}

\end{document}