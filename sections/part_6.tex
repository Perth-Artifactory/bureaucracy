\documentclass[../constitution.tex]{subfiles}
\begin{document}

\hypertarget{part-6-general-meetings-of-association}{%
\part{GENERAL MEETINGS OF ASSOCIATION}\label{part-6-general-meetings-of-association}}

\hypertarget{annual-general-meeting}{%
\subsection{Annual general meeting}\label{annual-general-meeting}}

\begin{enumerate}

\item The annual general meeting of the Association \chreplaced[id=proofing]{must}{will} take place within 3 months of the end of each financial year. \chcomment{Variation from the model rules, which allows for AGMs to be held later than 3 months after the end of a a financial year.}

\note {

Note for this subrule -

Notwithstanding that these rules require the annual general meeting to be held within 3 months of the end of each financial year, refer to section 50(3)(b) of the Act regarding notices which must be given to the Commissioner if the annual general meeting is not going to be held within 3 months.

}

\item The committee must determine the date, time and place of the annual general meeting.
\item The ordinary business of the annual general meeting is as follows ---

  \begin{enumerate}
  
  \item to confirm the minutes of the previous annual general meeting and of any special general meeting held since then if the minutes of that meeting have not yet been confirmed;
  \item to receive and consider ---

    \begin{enumerate}
    
    \item the committee's annual report on the Association's activities during the preceding financial year; and
    \item if the Association is a tier 1 association, the financial statements of the Association for the preceding financial year presented under Part 5 of the Act; and \label{consider-financial-statements-for-tier-1}
    \item if the Association is a tier 2 association or a tier 3 association, the financial report of the Association for the preceding financial year presented under Part 5 of the Act; \label{consider-financial-statements-for-tier-2-3}
    \item if required to be presented for consideration under Part 5 of the Act, a copy of the report of the review or auditor's report on the financial statements or financial report; \label{consider-auditors-report-financial-statements}
    \end{enumerate}
  \item to elect the committee members; \chcomment{Variation from the model rules - "To elect the office holders of the Association and other committee members".}
  \item if applicable, to appoint or remove a reviewer or auditor of the Association in accordance with the Act;
  \item to confirm or vary the entrance fees, subscriptions and other amounts (if any) to be paid by members.
  \end{enumerate}
\item Any other business of which notice has been given in accordance with these rules may be conducted at the annual general meeting.
\end{enumerate}

\note{

Note for this rule: 

Unless the Commissioner allows otherwise, under section 50(3) of the Act the annual general meeting must be held within 6 months after the end of the Association's financial year.  If it is the first annual general meeting, section 50(2) of the Act provides that it may be held at any time within 18 months after incorporation. 

}

\hypertarget{special-general-meetings}{%
\subsection{Special general meetings}\label{special-general-meetings}}

\begin{enumerate}

\item The committee may convene a special general meeting.
\item The committee must convene a special general meeting if at least 20\% of the members require a special general meeting to be convened.
\item The members requiring a special general meeting to be convened must ---

  \begin{enumerate}
  
  \item make the requirement by written notice given to the secretary; and \label{special-general-meeting-written-notice-to-secretary}
  \item state in the notice the business to be considered at the meeting; and
  \item each sign the notice.
  \end{enumerate}
\item The special general meeting must be convened within 28 days after notice is given under \chreplaced[id=proofing]{subrule}{rule} \ref{special-general-meeting-written-notice-to-secretary}
\item If the committee does not convene a special general meeting within that 28 day period, the members making the requirement (or any of them) may convene the special general meeting. \label{no-sgm-within-28-days-despite-written-request}
\item A special general meeting convened by members under \chreplaced[id=proofing]{subrule}{rule} \ref{no-sgm-within-28-days-despite-written-request} ---

  \begin{enumerate}
  
  \item must be held within 3 months after the date the original requirement was made; and
  \item may only consider the business stated in the notice by which the requirement was made.
  \end{enumerate}
\end{enumerate}

\hypertarget{notice-of-general-meetings}{%
\subsection{Notice of general meetings}\label{notice-of-general-meetings}}

\begin{enumerate}

\item The secretary or, in the case of a special general meeting convened under \chreplaced[id=proofing]{subrule}{rule} \ref{no-sgm-within-28-days-despite-written-request}, the members convening the meeting, must give to each member ---

  \begin{enumerate}
  
  \item at least 21 days' notice of a general meeting if a special resolution is to be proposed at the meeting; or
  \item at least 14 days' notice of a general meeting in any other case.
  \end{enumerate}
\item The notice must ---

  \begin{enumerate}
  
  \item specify the date, time and place of the meeting; and
  \item indicate the general nature of each item of business to be considered at the meeting; and
  \item if the meeting is the annual general meeting, include the names of any members who have nominated for election to the committee under rule \ref{notice-of-consideration-for-election-committee}; and
  \item if a special resolution is proposed ---

    \begin{enumerate}
    
    \item set out the wording of the proposed resolution as required by section 51(4) of the Act; and
    \item state that the resolution is intended to be proposed as a special resolution; and
    \item comply with rule \ref{notice-of-meeting-to-appointed-proxy}.
    \end{enumerate}
  \end{enumerate}

  \note{

  Note for this paragraph: 

  Section 51(1) of the Act states that a resolution is a special resolution if 
 it is passed — 

  (a) at a general meeting of an incorporated association; and 

  (b) by the votes of not less than three-fourths of the members of the association who cast a vote at the meeting. 

  }

\end{enumerate}

\hypertarget{proxies}{%
\subsection{Proxies}\label{proxies}}

\begin{enumerate}

\item Subject to \chreplaced[id=proofing]{subrule}{rule} \ref{proxy-for-no-more-than-5-other-members}, an ordinary member may appoint an individual who is an ordinary member \chcomment[id=RC]{not able to nominate committee members? (who become ordinary members at AGM?) LY - an ordinary member as defined in s1 means a member with voting rights.} as their proxy to vote and speak on their behalf at a general meeting.
\item An ordinary member may be appointed the proxy for not more than 5 other members. \label{proxy-for-no-more-than-5-other-members}
\item The appointment of a proxy must be in writing and signed by the member making the appointment.
\item The member appointing the proxy may give specific directions as to how the proxy is to vote on their behalf.
\item If no instructions are given to the proxy, the proxy may vote on behalf of the member in any matter as the proxy sees fit.
\item If the committee has approved a form for the appointment of a proxy, the member may use that form or any other form ---

  \begin{enumerate}
  
  \item that clearly identifies the person appointed as the member's proxy; and
  \item that has been signed by the member.
  \end{enumerate}
\item Notice of a general meeting given to an ordinary member under rule \chreplaced[id=proofing]{\ref{notice-of-general-meetings}}{50} must --- \label{notice-of-meeting-to-appointed-proxy}

  \begin{enumerate}
  
  \item state that the member may appoint an individual who is an ordinary member as a proxy for the meeting; and
  \item include a copy of any form that the committee has approved for the appointment of a proxy.
  \end{enumerate}
\item A form appointing a proxy must be given to the secretary before the commencement of the general meeting for which the proxy is appointed. \label{form-appointing-proxy-before-general-meeting}
\item A form appointing a proxy sent by post or electronically is of no effect unless it is received by the Association not later than 24 hours before the commencement of the meeting.
\end{enumerate}

\hypertarget{use-of-technology-to-be-present-at-general-meetings}{%
\subsection{Use of technology to be present at general meetings}\label{use-of-technology-to-be-present-at-general-meetings}}

\begin{enumerate}

\item Participation in meetings via electronic communications technology shall be considered attendance if acceptable to the Secretary and Chair. \label{electronic-participation-general-meeting-acceptable-to-secretary-and-chair} \chcomment{This clause varies from the Model Rules for no really good reason.}
\item A member who participates in a general meeting as allowed under \chreplaced[id=proofing]{subrule}{rule} \ref{electronic-participation-general-meeting-acceptable-to-secretary-and-chair} is taken to be present at the meeting and, if the member votes at the meeting, the member is taken to have voted in person.
\end{enumerate}

\hypertarget{presiding-member-and-quorum-for-general-meetings}{%
\subsection{Presiding member and quorum for general meetings}\label{presiding-member-and-quorum-for-general-meetings}}

\begin{enumerate}

\item Quorum for a committee meeting is stated in Guidance Note E. \chcomment{Revise constitution to clarify what "Guidance Notes" are intended to refer to.}
\item The chairperson or, in the chairperson's absence, the deputy chairperson must preside as chairperson of each general meeting.
\item If the chairperson and deputy chairperson are absent or are unwilling to act as chairperson of a general meeting, the committee members at the meeting must choose one of them to act as chairperson of the meeting.
\item No business is to be conducted at a general meeting unless a quorum is present.
\item If a quorum is not present within 30 minutes after the notified commencement time of a general meeting ---

  \begin{enumerate}
  
  \item the members present may elect to wait up to another 60 minutes for the quorum to be present; or \chcomment{Variance to the model rules, which have no such clause. Suggest deleting this clause.}
  \item in the case of a special general meeting --- the meeting lapses; or
  \item in the case of the annual general meeting --- the meeting is adjourned to --- \label{adjourned-annual-general-meeting}

    \begin{enumerate}
    
    \item the same time and day in the following week; and
    \item the same place, unless the chairperson specifies another place at the time of the adjournment or written notice of another place is given to the members before the day to which the meeting is adjourned.
    \end{enumerate}
  \end{enumerate}
\item If ---

  \begin{enumerate}
  
  \item a quorum is not present within 30 minutes after the commencement time of an annual general meeting held under \chreplaced[id=proofing]{subrule}{rule} \ref{adjourned-annual-general-meeting}; and
  \item the \chreplaced[id=proofing]{lesser}{lessor} of 10 or 10\% ordinary members are present at the meeting, \chcomment{Variance to model rules - "at least 2 ordinary members are present at the meeting," }
  \end{enumerate}
  those members present are taken to constitute a quorum.
\end{enumerate}

\hypertarget{adjournment-of-general-meeting}{%
\subsection{Adjournment of general meeting}\label{adjournment-of-general-meeting}}

\begin{enumerate}

\item The chairperson of a general meeting at which a quorum is present may, with the consent of a majority of the ordinary members present at the meeting, adjourn the meeting to another time at the same place or at another place. \label{chairperson-may-adjourn-meeting-with-consent}
\item Without limiting \chreplaced[id=proofing]{subrule}{rule} \ref{chairperson-may-adjourn-meeting-with-consent}, a meeting may be adjourned ---

  \begin{enumerate}
  
  \item if there is insufficient time to deal with the business at hand; or
  \item to give the members more time to consider an item of business.
  \end{enumerate}
\item No business may be conducted on the resumption of an adjourned meeting other than the business that remained unfinished when the meeting was adjourned.
\item Notice of the adjournment of a meeting under this rule is not required unless the meeting is adjourned for 14 days or more, in which case notice of the meeting must be given in accordance with rule \ref{notice-of-general-meetings}.
\end{enumerate}

\hypertarget{voting-at-general-meeting}{%
\subsection{Voting at general meeting}\label{voting-at-general-meeting}}

\begin{enumerate}

\item On any question arising at a general meeting ---

\chcomment{Variance from the model rules - The provisions for voting by members who are body corporates have been deleted. Should add these back so that we can have corporate members, i.e. corporate sponsors. Subrules (2) through (4) and mods to subrule (8).}

  \begin{enumerate}
  
  \item subject to \chreplaced[id=proofing]{subrule}{rule} \ref{chairperson-has-casting-vote-if-votes-tied}, each ordinary member has one vote; and
  \item ordinary members may vote personally or by proxy.
  \end{enumerate}
\item Except in the case of a special resolution, a motion is carried if a majority of the ordinary members present at a general meeting vote in favour of the motion.
\item If votes are divided equally on a question, the chairperson of the meeting has a second or casting vote. \label{chairperson-has-casting-vote-if-votes-tied}
\item If the question is whether or not to confirm the minutes of a previous general meeting, only members who were present at that meeting may vote.
\item For a person to be eligible to vote at a general meeting as an ordinary member, they

  \begin{enumerate}
  
  \item must have been an ordinary member at the time notice of the meeting was given under rule \ref{notice-of-general-meetings}; and
  \item must have paid any fee or other money payable to the Association by the member. \chcomment{Subject of a CAP.}
  \end{enumerate}
\end{enumerate}

\hypertarget{when-special-resolutions-are-required}{%
\subsection{When special resolutions are required}\label{when-special-resolutions-are-required}}

\begin{enumerate}

\item A special resolution is required if it is proposed at a general meeting --- \label{special-resolution-required-when}

  \begin{enumerate}
  
  \item to affiliate the Association with another body; or
  \item to request the Commissioner to apply to the State Administrative Tribunal under section 109 of the Act for the appointment of a statutory manager.
  \end{enumerate}
\item \chreplaced[id=proofing]{Subrule}{rule} \ref{special-resolution-required-when} does not limit the matters in relation to which a special resolution may be proposed.

\note{

Note for this rule: 

Under the Act, a special resolution is required if an incorporated association proposes to do any of the following — 

(a) to adopt (the) model rules (section 29(1)); 

(b) to alter its rules, including changing the name of the association (section 30(1)); 

(c) to decide to apply for registration or incorporation as a prescribed body corporate (section 93(1)); 

(d) to approve the terms of an amalgamation with one or more other incorporated associations (section 102(4)); 

(e) to be wound up voluntarily (section 121(2)) or by the Supreme Court (section 124(a) and Schedule 4 item 9); 

(f) to cancel its incorporation (section 129). 

}

\end{enumerate}

\hypertarget{determining-whether-resolution-carried}{%
\subsection{Determining whether resolution carried}\label{determining-whether-resolution-carried}}

\begin{enumerate}

\item In this rule ---
  
  \textbf{poll} means the process of voting in relation to a matter that is conducted in writing.

\item Subject to \chreplaced[id=proofing]{subrule}{rule} \ref{if-poll-on-any-question-is-demanded}, the chairperson of a general meeting may, on the basis of general agreement or disagreement or by a show of hands, declare that a resolution has been --- \label{chairperson-declare-resolution-based-on-show-of-hands}

  \begin{enumerate}
  
  \item carried; or
  \item carried unanimously; or
  \item carried by a particular majority; or
  \item lost.
  \end{enumerate}
\item If the resolution is a special resolution, the declaration under \chreplaced[id=proofing]{subrule}{rule} \ref{chairperson-declare-resolution-based-on-show-of-hands} must identify the resolution as a special resolution.
\item If a poll is demanded on any question by the chairperson of the meeting or by at least 3 other ordinary members present in person or by proxy --- \label{if-poll-on-any-question-is-demanded}

  \begin{enumerate}
  
  \item the poll must be taken at the meeting in the manner determined by the chairperson;
  \item the chairperson must declare the determination of the resolution on the basis of the poll.
  \end{enumerate}
\item If a poll is demanded on the election of the committee \chcomment{Variance from the model rules, which state "chairman", not "committee".} or on a question of an adjournment, the poll must be taken immediately.
\item If a poll is demanded on any other question, the poll must be taken before the close of the meeting at a time determined by the chairperson.
\item A declaration under \chreplaced[id=proofing]{subrule}{rule} \ref{chairperson-declare-resolution-based-on-show-of-hands} or \ref{if-poll-on-any-question-is-demanded} must be entered in the minutes of the meeting, and the entry is, without proof of the voting in relation to the resolution, evidence of how the resolution was determined.
\end{enumerate}

\hypertarget{minutes-of-general-meeting}{%
\subsection{Minutes of general meeting}\label{minutes-of-general-meeting}}

\begin{enumerate}

\item The secretary, or a person authorised by the committee from time to time, must take and keep minutes of each general meeting.
\item The minutes must record the business considered at the meeting, any resolution on which a vote is taken and the result of the vote.
\item In addition, the minutes of each annual general meeting must record ---

  \begin{enumerate}
  
  \item the names of the ordinary members attending the meeting; and
  \item any proxy forms given to the Secretary of the meeting under rule \ref{form-appointing-proxy-before-general-meeting}; and
  \item the financial statements or financial report presented at the meeting, as referred to in rule \ref{consider-financial-statements-for-tier-1} or \ref{consider-financial-statements-for-tier-2-3}; and
  \item any report of the review or auditor's report on the financial statements or financial report presented at the meeting, as referred to in rule \ref{consider-auditors-report-financial-statements}.
  \end{enumerate}
\item The minutes of a general meeting must be entered in the Association's minute record within 7 days \chcomment{Variance to model rules - 30 days} after the meeting is held.
\item The chairperson must ensure that the minutes of a general meeting are reviewed as correct by a motion of the next general meeting. \chcomment{Variance to model rules - review.  (5) The chairperson must ensure that the minutes of a general meeting are reviewed and signed as correct by —   (a) the chairperson of the meeting; or  (b) the chairperson of the next general meeting. }
\item When the minutes of a general meeting have been signed \chcomment{As per rule above, miinutes are never "signed"} as correct they are, in the absence of evidence to the contrary, taken to be proof that ---

  \begin{enumerate}
  
  \item the meeting to which the minutes relate was duly convened and held; and
  \item the matters recorded as having taken place at the meeting took place as recorded; and
  \item any election or appointment purportedly made at the meeting was validly made.
  \end{enumerate}
\end{enumerate}

\end{document}