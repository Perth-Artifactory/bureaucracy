\documentclass[../constitution.tex]{subfiles}
\begin{document}

\part*{Preface}

\note{
    This preface is informative only, and is not part of the rules of the Association.

    This preface will be deleted before the revised rules are submitted to the Commissioner.
}

The rules of the Association have been revised as follows.

The changes to the rules include:

\begin{itemize}
    \item Proofing
    \item CAP1
    \item CAP2
    \item CAP3
\end{itemize}

Background

The rules of the Association are based on the model rules, with modifications 


\section*{What has been changed?}

\subsection*{Proofing}

The current rules of the Association vary from the Model Rules in many ways.




The rules of the Association have been revised so that ---

\begin{itemize}
    \item The wording, the terms used, punctuation, capitalization, and formatting all match the Model Rules as closely as possible, except where a variation is genuinely intended.
    \item Proofreading in general.
    \item Minor ambiguities corrected. E.g. membership class swapping.
    \item The explanatory notes from the Model Rules have been added into the document.
\end{itemize}

The rules of the Association were compared to the Model Rules and the 

\subsection*{CAP1-Exclude-Voluntary-Invoices}

Rule \ref{voting-at-general-meeting} regarding \textit{Voting at general meeting} currently states that a member can only vote at a general meeting if they have "...paid any fee or other money payable to the Association by the member".

A literal reading of this rule implies that a member who has an outstanding invoice for a voluntary pledge to the Association (e.g. pledgeBot) would be unable to vote, even if the invoice was not overdue.

Rule \ref{voting-at-general-meeting} has been revised to specifically require that a member who wants to vote at a general meeting ---

\begin{itemize}
    \item have paid their membership fees, up to the date of the meeting; and
    \item must not have any other \textbf{overdue} debts, except pledged donations.
\end{itemize}




\subsection*{CAP3 - Community Shed}

Changes marked \chadded[id=CAP3-Community-Shed]{thus} modify the rules as required so that the Association is eligible for deductible gift recpient (DGR) status as a "community shed".

Community sheds, also known as men's sheds or women's sheds, can be broadly defined as places where people can go to meet new people, socialise, and share skills and knowledge in a workshop environment.

The Artifactory already does many of the same things as a community shed, and provides many of the same services and benefits to the community. Thus it is logical that the Association should apply for deductible gift recipient status as a community shed.

The proposed amendments to the rules of the Association are required to meet the legal requirements for the Association to be a DGR in the community shed category, under the Income Tax Assessment Act 1997.

* The purpose and objects are modified in order to meet the legal definition
* Membership applications to meet open membership.
* Winding up clause.

The benefit of becoming a DGR is that donations to the Association become tax-deductible, which means that donations to the Association will go further.


A few other changes to membership have also been lumped in here.

The restriction that all applicants must be a "person at least 18 years of age" has been deleted. This would have made it impossible for us to admit junior members, e.g. first-year university students. The rules have been revised so that persons under the age of 18 can be members, so long as they are in a class of membership that does not confer full voting rights.

Ed: Do we want to restrict, in the definition of the classes of membership or in the rules, a minimum age (say 15) to be a junior member? e.g. to prevent 10 year olds signing up (which we would otherwise be obliged to accept.) e.g. a rule which states "An individual who has not reached the age of 18 years may only apply for a class of membership which explicitly allows applications from persons under 18 years of age."




The Income Tax Assessment Act 1997 (Cth) defines ---


community shed means a public institution that satisfies all of the following requirements:

	(a)	the institution's dominant purposes are advancing mental health and preventing or relieving social isolation;

	(b)	the institution seeks to achieve those purposes principally by providing a physical location where it supports individuals to undertake activities, or work on projects, in the company of others;

	(c)	either:

	(i)	there are no particular criteria for membership of the institution; or

	(ii)	the criteria for membership of the institution relate only to an individual's gender or Indigenous status (in that membership is, for cultural reasons, open only to *Indigenous persons) or both.


\subsection*{CAP4-Corporate-Members}

The Model Rules provide that body corporates can be members of the Association, but these provisions had been deleted from the current rules.

The provisions for body corporates to be members have been reinstated.

This allows the Association to admit e.g. corporate sponsors as a class of membership.







What next?

This proposal for revising the rules of the Association will be open for comments from XXX to XXX.

Comments will be incorporated.

Motions to be raised at the general meeting will be:

Accept the revised rules of the Association in their entirety.

If this motion fails, then motions will be raised for each of:

Accept the revised rules marked "proofing".

Accept the revised rules marked "CAP3-Community-Shed".

Note special resolutions, note requirements of act relating to revision of rules.



\end{document}