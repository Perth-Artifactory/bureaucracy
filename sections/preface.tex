\documentclass[../constitution.tex]{subfiles}
\begin{document}

\part*{Preface: Proposed (Draft) amendments to these rules}

\note{
    This preface is informative only, and is not part of the rules of the Association.

    This preface will be deleted before the revised rules are submitted to the Commissioner.
}

The proposed amendments to the rules of the Association are explained in this section.

Changes to the text are marked like \chadded[id=proofing]{this} or \chdeleted[id=CAP1-Exclude-Voluntary-Invoices]{this}.

Additionally, rule \ref{emergency-disciplinary-actions} regarding \textit{Emergency temporary suspension} has been completely rewritten and is now called \textit{Emergency disciplinary actions}.

Comments, which are not part of the text, appear similar to \chcomment{this.}

\section*{Proofing}

The rules have been proofread and checked as follows.

\begin{itemize}
    \item The wording, the terms used, punctuation, capitalization, and formatting have been conformed to the \href{https://www.commerce.wa.gov.au/consumer-protection/model-rules}{model rules} as closely as possible, except where a variation is genuinely intended.
    \item Spelling, grammar, etc. corrected.
    \item Minor ambiguities corrected. E.g. membership class swapping in subrule \ref{swap-ordinary-associate-membership}.
    \item The explanatory notes from the Model Rules are included in the document.
\end{itemize}

\section*{CAP1-Exclude-Voluntary-Invoices}

This is an amendment to subrule \ref{member-cannot-vote-with-debts} regarding eligibility to vote at general meetings.

This subrule previously required that a member who wanted to vote at a general meeting "must have paid any fee or other money payable to the Association by the member."

At face value, this would mean that having any outstanding invoice would prevent a member from voting, regardless of what the invoice related to, or if the invoice was overdue or not.

The subrule has been amended so that members who want to vote ---

\begin{itemize}
    \item Must have paid their membership fees; and
    \item Must not have any \textbf{overdue} debts; except where those debts relate to invoices for voluntary donations.
\end{itemize}

\section*{CAP2-Permissive-Written-Notice-For-Committee}

This is an amendment to rule \ref{nomination-of-committee-members} regarding \textit{Nomination of committee members}.

This rule currently requires nominations for committee members to be sent via written notice to the secretary. Written notice would usually be interpreted as a hard copy document or an email, as per rule \ref{giving-notices-to-members}, and would not include Slack messages.

The amendment changes rule \ref{nomination-of-committee-members} to allow more flexible means of submitting nominations to the secretary, which may include (for example) Slack direct messages.

\section*{CAP3-Community-Shed}

This is a set of amendments to enable the Association to apply for deductible gift recipient (DGR) status as a "community shed".

Community sheds, including men's sheds and women's sheds, can be informally defined as "workshops where people can meet people, socialise, and share skills and knowledge." The Artifactory already meets this informal definition.

In fact, the Artifactory offers much more to the Perth community than a typical community shed, by ---

\begin{itemize}
    \item Being open to, and truly welcoming, members of all ages, genders, and orientations; and
    \item Engaging with younger demographics, including university students, not traditionally served by typical community sheds; and
    \item Offering a breadth of tools, training, and services well beyond the scope of typical community sheds.
\end{itemize}

As such, it makes sense for the Artifactory to take the steps required to obtain official deductible gift recipient status as a community shed.

DGR status will enable us to better fulfil our objects and purposes, thus delivering more benefits to the Perth community.

\subsection*{Requirements for DGR eligibility as a community shed}

A deductible gift recipient (DGR) is an organisation or fund that registers to recieve tax deductible gifts or donations. \footnote{\href{https://www.ato.gov.au/individuals/income-and-deductions/deductions-you-can-claim/other-deductions/gifts-and-donations/}{ATO - Gifts and donations}}

\bigskip

Community sheds are a new category of DGR, introduced in 2020.

\begin{itemize}
    \item The eligbility criteria for DGRs generally, and community sheds specifically, are defined in the \textit{Income Tax Assessment Act (Cth) 1997}.
    \item The Australian Tax Office (ATO) publishes a \href{https://www.ato.gov.au/Non-profit/Getting-started/In-detail/Types-of-DGRs/Community-sheds/?=redirected_communitysheds}{webpage} and a \href{https://cdn2.app.viostream.com/3da28d19-5792-4397-b158-a4740090a8d8/d1a1ccce-1c54-45f0-9f7e-9a8f63d2d05b/d74a313a-96c3-418e-9390-ac4600afbf00.pdf?response-content-disposition=attachment%3bfilename%3d2020+Help+Sheet+_+Community+Sheds+_+Meeting+DGR+endorsement+_20200921.pdf}{help sheet} regarding community sheds.
\end{itemize}

\bigskip

As per the help sheet, the criteria are:

\begin{itemize}
    \item Have an active ABN - Yes - \href{https://abr.business.gov.au/ABN/View/16847853023}{16 847 853 023}.
    \item Be a registered charity with ACNC - Yes - \href{https://www.acnc.gov.au/charity/charities/705391a6-3aaf-e811-a963-000d3ad24077/profile}{ACNC profile}.
    \item Be established and operate in Australia - Yes.
    \item Meet the \textbf{DGR category requirements} for the "community shed" category ---
    \begin{itemize}
        \item A public institution whose \textbf{dominant purposes} are advancing mental health and preventing or relieving social isolation.
        
        We are already a public institution, specifically an incorporated association under the \textit{Associations Incorporation Act 2015 (WA)}.
        
        Our activities, including our regularly scheduled general workshop days, electronics workshops, metal-working workshops, and general community organisation, already have the effect of \textit{"advancing mental health and preventing or relieving social isolation"}.
        
        We will formally address this criteria by changing our purpose and objects.
        \item Principally advances these purposes through providing a \textbf{physical location} and supports individuals to work on projects or undertake other activities in the company of others.
        
        We already meet this criteria.
        \item Has \textbf{membership} that is open, or is limited only to an individual's gender or indigenous status or both.
        
        Addressed by modification to our membership application process, so that membership will be open to any individual over the age of 18 \textit{"who supports the objects or purposes of the Association"}.
    \end{itemize}
    \item Include \textbf{DGR winding up and revocation clauses} in your governing documents.
    
    Addressed by adding to our existing winding-up clause.
\end{itemize}

\subsection*{Purposes and objects}

Our purposes and objects, at Key Information - Item \ref{key-info-objects-purposes}, are modified to include:

\begin{displayquote}
The dominant purpose of the Association is to advance mental health and prevent or relieve social isolation.

The Association pursues this purpose by encouraging and facilitating creative use of technology at the Association's physical premises, where the Association supports individuals to undertake activities, or work on projects, in the company of others.
  
The objects of the Association, which fulfil its purpose, are to: ...
\end{displayquote}

The remainder of our objects remain the same, reflecting that we will deliver a unique kind of community shed to the Perth community through our focus on "the creative use of technology", "artistic and technological projects", and so on.

\subsection*{Open membership}

The ATO states that ---

\begin{displayquote}

    A community shed must be open to the community to join and generally not impose criteria restricting membership based on matters such as age, ethnicity or background.

    Rejecting an application for arbitrary reasons will not constitute open membership.

    \dots

    To meet the open membership requirements, you must have a policy and process in place that clearly demonstrates all new members are nominated and approved without exception. This should be reflected in your governing rules.

\end{displayquote}

\bigskip

To meet this requirement,

\begin{itemize}
    \item The committee no longer has the power to reject a membership application for arbitrary reasons.
    \item Instead, the committee may only reject a membership application for the specific reasons listed in subrule \ref{reasons-why-committee-can-reject-membership-application}.
    \item Subrule \ref{reject-membership-detrimental} does allow for the committee to reject applicants who have behaved in a way that is detrimental to the interests of the Association - e.g. against our Code of Conduct.
\end{itemize}

\bigskip

Given that we are changing from a model where the committee assesses applicants individually, to a model where the committee must accept all applicants automatically, we will also need to consider how we control:

\begin{itemize}
    \item access to the space outside of scheduled sessions, e.g. issue of 24/7 access keys; and
    \item access to high-risk tools in the space; and
    \item inductions to the space generally.
\end{itemize}

\bigskip

We may need to implement changes to our by-laws and our website in order to effectively manage the risks associated with new members.

These may include requring a certain level of attendance at regular scheduled events, and completion of prescribed training/inductions, before granting after-hours access.

This would be in line with requirements of other associations (e.g. sports shooting clubs) which are implemented to fulfil our duty of care with regards to health and safety of all participants in the space.

\bigskip

Also included in this section are changes to membership eligibility for under-18's.

\begin{itemize}
    \item The restriction that all applicants must be a "person at least 18 years of age" has been deleted. This currently makes it impossible for us to admit junior members, e.g. first-year university students.

    \item The rules have been revised so that persons under the age of 18 can be members, so long as they are in a class of membership that explicitly allows under-18's (e.g. "Student Member") and the class of membership does not confer full voting rights.
\end{itemize}

\note {
    We should consider using the next general meeting as an opportunity to create a grade of (under 18) student membership.
}

\subsection*{Winding-up clause}

Rule \ref{distribution-of-surplus-property-on-cancellation-of-incorporation-or-winding-up} has been amended to include the language recommended by the ATO in relation to distribution of assets if the Association winds up or loses its DGR status.


\section*{CAP4-Corporate-Members}

These amendments restore the provision for body corporates to be members of the Association.

These provisions are part of the Model Rules that were deleted when the model rules were first adapted for the Artifactory.

Restoring these provisions allows the Association to admit e.g. corporate sponsors as a class of membership, broadening our options for raising funds and generating community involvement.

Subrule \ref{reject-membership-body-corporate} allows the committee to reject membership applications from body corporates at will.

\section*{CAP5-Meeting-At-More-Than-One-Place}

These amendments replace the provisions for meetings to be held electronically "if acceptable to the Secretary and the Chair" with more robust language, taken from the constitution of the West Australian Cricket Association.

For committee meetings, rule \ref{notice-of-committee-meetings} and rule \ref{use-of-technology-to-be-present-at-committee-meetings} were revised.

For general meetings, rule \ref{notice-of-general-meetings} and rule \ref{use-of-technology-to-be-present-at-general-meetings} were revised.

\section*{CAP6-Delete-Further-Delay-Clauses}

Rule \ref{quorum-for-committee-meetings} for committee meetings, and rule \ref{presiding-member-and-quorum-for-general-meetings} for general meetings, were revised to delete a clause which allowed for members to wait an additional 60 minutes for members to arrive at a meeting, after the 30 minutes allowed for by the model rules.

The application of the "extra 60 minutes" clause in relation to the following clauses was unclear. Additionally, it was not clear what risk the extra clause addressed, which was not already addressed by the model rules.

\section*{CAP7-Membership-Fees}

Rule \ref{membership-fees} was clarified regarding:

Setting entrance and membership fees ---

\begin{itemize}
    \item The \textbf{amount} of the fees is set by resolution at a general meeting.
    \item The \textbf{schedule} for paying the fees (e.g. yearly, monthly) is set by the committee.
\end{itemize}

The procedure for re-admitting a member whose membership had lapsed under subrule \ref{member-fails-to-pay-fee}.

\section*{CAP8-Common-Seal}

This amendment adds a simple statement to Rule \ref{executing-documents} stating that the Association does not have a common seal, as recommended by \href{https://www.commerce.wa.gov.au/publications/whats-rules}{Commerce WA}.

\section*{CAP9-Emergency-Disciplinary-Action}

Rule \ref{emergency-temporary-suspension} for \textit{Emergency temporary suspension} has been changed to \textit{Emergency disciplinary actions} (\textbf{EDAs}) and re-drafted.

This rule gives the committee the power to discipline members with \textbf{immediate} effect, where necessary to protect the Association and its members.

This contrasts with the full \textit{Suspension and expulsion} process (rule \ref{suspension-or-expulsion}) which takes at least 28 days to implement.

The re-drafted version ---

\begin{itemize}
    \item States that EDAs may be considered where a member "has acted against the interests of the Association" - the same language used in rule \ref{suspension-or-expulsion} regarding \textit{Suspension or expulsion}.
    \item Still requires two committee members to start an EDA.
    \item Is more specific about the requirements for the initial notice and the written notice.
    \item Deletes the section relating to appeals, since this is already covered under the general grievance procedure and the mediation procedure.
    \item Lists specific kinds of restrictions which are contemplated for EDAs, rather than broadly "temporarily modify member rights".
    \item Adds a statement that members are not entitled to refunds etc. during the period of an EDA.
\end{itemize}

\section*{CAP10-Committee-Election}

These amendments clarify how committee members are elected, and how the committee members elect the office holders of the Association.

\begin{itemize}


\item Rule \ref{committee-members} \textit{Committee members} - delete the term "Executive committee", which is not used anywhere else in the rules or the by-laws.

\item Rule \ref{election-of-committee-members} \textit{Election of committee members} - clarify that this is about electing committee members, not ordinary committee members specifically.

Also delete the redundant clause stating that proxy votes are permitted - already covered under rule \ref{proxies} regarding \textit{Proxies}.

\item Rule \ref{election-of-office-holders} \textit{Election of office holders} - clarify how the election of office holders works:

  \begin{itemize}
  \item Define that the committee members must meet \textit{as soon as practicable} after the AGM. (Previously the timeframe was not defined.)
  \item Define who is the chairperson of the initial committee meeting. (Previously undefined.)
  \item Define the ordinary business of the meeting. (Previously undefined.)
  \end{itemize}

\end{itemize}

\section*{CAP11-Payments-To-Committee-Members}

As it stands, rule \ref{payments-to-committee-members} regarding \textit{Payments to Committee Members} prohibits any payment to committee members for \textbf{out-of-pocket expenses for travel and accommodation} in connection with the Association's business.

This is the reverse of the model rules, which states that committee members \textbf{are} allowed to be reimbursed for travel expenses.

The likely intent of the current rule is that committee members of the Artifactory, who usually live in the Perth metropolitan area, should not be allowed to claim expenses for travelling within Perth.

However the complete prohibition against paying for expenses that are "otherwise in connection with the Association's business" also restricts committee members from being reimbursed for travel that is genuinely helpful to the Artifactory.

For example, we might want to send a committee member to a client's site to deliver services, as part of a commercial agreement between the Artifactory and a third party, where the expenses will be paid out of the Artifactory's fee for those services.

The proposed amendment to the rules allows for committee members to be reimbursed for out-of-pocket travel and accommodation expenses so long as ---

\begin{itemize}
    \item The expenses do not relate to attendance at committee meetings or general meetings; and
    \item The expenses are approved in principle by the committee before the expenses are incurred.
\end{itemize}


\section*{CAP12-Multiple-Bank-Accounts}

Rule \ref{control-of-funds} regarding \textit{Control of funds} refers to the Association having a single bank account.

The Association needs to have multiple bank accounts, to allow for separation of funds from different sources and for different purposes.

The language of rule \ref{control-of-funds} is amended to allow for the Association to have multiple bank accounts.

\part*{What next?}

These proposed amendments to the rules are open for comment until Friday 2023-02-03.

\begin{itemize}
    \item Informal discussion may be conducted on Slack, in the channel \href{https://perthartifactory.slack.com/archives/C04GDNBFGN5}{\#governance}.
    \item Formal submissions of comments, to be considered by the committee, \textbf{must be sent to the secretary} - secretary@artifactory.org.au - by \textbf{5pm AWST on Friday 2023-02-03.}
\end{itemize}

\bigskip

Once the proposed amendments are finalised, the secretary will convene a \textit{special general meeting} to vote on the amendments.

We currently expect to hold the special general meeting on Saturday 2023-03-04.

Formal notice of this meeting will be sent to all members of the Association in due course.

\note{
Changes to the rules of the Association must pass as \textit{special resolutions} at a general meeting.

* Special resolutions have \href{https://www.commerce.wa.gov.au/books/inc-guide-incorporated-associations-western-australia/meeting-procedures}{specific notice requirements} and cannot be amended at the meeting.

* Special resolutions must pass \textit{"by the votes of not less than three-fourths of the members of the Association who cast a vote at the meeting"}.
}

\bigskip

\large

\textcolor{red}{Because of specific notice requirements mandated by the Act, it is \textbf{impossible} to alter a special resolution, or alter the proposed amendments to the rules, during a meeting.}

\textcolor{red}{Therefore, it will \textbf{not} be possible or appropriate to perform any review of the amendments in the general meeting.}

\textcolor{red}{Members who want their comments to be considered \textbf{must} submit those comments by \textbf{5pm AWST on Friday 2023-02-03.}}

\end{document}